\chapter{插件的实现}

本章主要介绍面向Hyperledger Fabric的智能合约开发插件的实现细节。
基于概要设计和详细设计,阐述Fabric智能合约初始模板生成、代码智能提示、模拟调用及测试三个功能模块的技术实现。
同时,本章将对部分功能的实现结果进行展示。

\section{Fabric智能合约初始模板生成模块实现}

\subsection{生成器用户接口实现}

由需求分析和系统设计可知,此模块旨在为智能合约开发者生成一份初始的Fabric智能合约。
因此,插件本质上需要实现一个代码生成器,该代码生成器的目标对象是Fabric智能合约。

插件基于IntelliJ平台提供的DevKit SDK进行开发,其中包含的Action系统可以支持生成器实现用户接口层,代码结构如图~\ref{fig:5.1}所示。

\begin{figure}[htb]
  \centering
  \includegraphics[width=0.65\linewidth]{figure/implement/生成器代码结构.pdf}
  \caption{生成器用户接口层代码结构}\label{fig:5.1}
\end{figure}

Generator作为自定义Action需要继承AnAction作为父类,调用父类构造函数以定义入口按钮的名称,并重写父类方法\texttt{actionPerformed()}实现代码生成器的业务逻辑。
在业务逻辑中通过调用参数AnActionEvent的\texttt{getProject()}方法可以获取当前工作空间的项目。

为了使得自定义的Action在IntelliJ IDE中生效,需要在配置文件中注册该Action,配置代码如图~\ref{fig:5.2}所示。

\begin{figure}[htb]
  \centering
  \includegraphics[width=\linewidth]{figure/implement/Action注册实现.pdf}
  \caption{Action注册实现}\label{fig:5.2}
\end{figure}

如上图所示,为了使插件具有良好的用户体验感,系统为智能合约开发者提供了代码生成器的多个入口,包括工具菜单栏、代码行右键的生成按钮、帮助菜单中的搜索框以及快捷键Ctrl+Shift+G。
配置文件中自定义Action的各个配置项描述如表~\ref{table:generateXml}所示。

\begin{table}[htb]\scriptsize
\centering
\caption{Action配置项描述}
\vspace{2mm}
% l - left, r - right, c - center. | means one vertical line
\begin{tabular}{cc}
\toprule
\textbf{属性}&\textbf{描述}\\
\midrule
\multirow{2}*{\textbf{Action ID}}&Action在插件中的特殊标识符,若为一个Action提供多个入口,\\
~&则需要不同ID,例如在ID基础上添加后缀\\ \hline
\multirow{2}*{\textbf{Class}}&插件Action的实现类,需要继承AnAction,\\
~&可以被多个Action重用\\ \hline
\textbf{Text}&插件Action在菜单中显示的文本\\ \hline
\textbf{Description}&插件Action设置的简要描述信息\\ \hline
\textbf{Icon}&插件Action图标路径\\ \hline
\multirow{3}*{\textbf{Add-to-group}}&Group-id标识着插件Action加入的Action组,\\
~&指定了Action在IDE中的位置;Anchor代表Action锚点,\\ 
~&设置在Action Group中该Action相对其他Action的位置\\ \hline
\textbf{Keyboard-shortcut}&定义了插件Action的快捷键\\
\bottomrule
\end{tabular}
\label{table:generateXml}
\end{table}

该模块依赖IntelliJ平台的用户接口组件Messages实现了良好的用户交互。
Messages提供了多种IntelliJ IDE弹窗,包括带有文本框的输入弹窗、带有组合框的选择弹窗以及多种信息弹窗。

插件根据不同需求场景采用了对应的弹窗进行用户交互。
开发者当前的工作空间存在项目是生成智能合约初始文件的前提,因此,如果插件检测到没有任何项目存在,就会弹出错误窗口,提醒开发者必须创建一个新项目;
插件通过输入弹窗让开发者可以设置智能合约的名称;
插件通过选择弹窗让开发者设置智能合约的开发模式;
最后,插件通过信息弹窗通知开发者智能合约初始模板的创建状态。
代码片段如图~\ref{fig:5.3}所示。

\begin{figure}[htb]
  \centering
  \includegraphics[width=0.8\linewidth]{figure/implement/用户交互弹窗实现.pdf}
  \caption{用户交互弹窗实现}\label{fig:5.3}
\end{figure}

\subsection{自定义Action实现}

模块在自定义Action内部实现Fabric智能合约生成器的业务逻辑,该过程基于FreeMarker模板技术以及Java IO文件操作。
Fabric智能合约生成器首先通过用户接口层获取开发者设置的智能合约名称以及选择的智能合约开发模式,然后基于获取到的数据构建FTL模板的数据模型(DataModel),FreeMarker引擎加载模板并处理数据模型,最后利用Java IO创建智能合约文件并将模板内容写入文件。
Fabric智能合约生成器的业务逻辑代码实现片段如图~\ref{fig:5.4}所示。

\begin{figure}[htb]
  \centering
  \includegraphics[width=0.8\linewidth]{figure/implement/Fabric智能合约生成器业务逻辑.pdf}
  \caption{Fabric智能合约生成器实现}\label{fig:5.4}
\end{figure}

插件封装FreeMarker模板提供的服务并实现了模板工具类TemplateUtil。
Fabric智能合约生成器将调用TemplateUtil提供的静态方法完成对FreeMarker引擎的配置和模板的加载,处理数据模型并输出,TemplateUtil代码实现片段如图~\ref{fig:5.5}所示。

\begin{figure}[htb]
  \centering
  \includegraphics[width=\linewidth]{figure/implement/工具类TemplateUtil实现.pdf}
  \caption{工具类TemplateUtil实现}\label{fig:5.5}
\end{figure}

FreeMarker模板引擎可以通过内建模板加载器加载指定文件,调用\texttt{setClassForTemplateLoading()}方法设置类文件路径,或调用~\texttt{setDir-} \linebreak ~\texttt{ectioryForTemplateLoading()}方法设置文件的绝对路径,还可以基于Web服务器的路径加载模板。
结合需求分析并考虑到本插件不仅限于本地使用,因此模块采用类路径加载方式,将FTL模板放在系统的resources资源中。

FreeMarker模板引擎的配置类Configuration需要确定引擎的版本从而完成实例化,调用Configuration的\texttt{setTemplateExceptionHandler()}方法设置引擎的异常处理机制。

FreeMarker模板引擎提供了常见场景下采用的几种异常处理机制,如表~\ref{table:exception}所示。
由于插件在自定义Action中需要针对不同d 异常情况做相应的处理,因此设置引擎的异常处理器为RETHROW\_HANDLER,当引擎运行出错时不进行任何处理直接重新抛出错误,将控制权交还给插件本身。

\begin{table}[htb]\scriptsize
\centering
\caption{FreeMarker引擎异常处理机制}
\vspace{2mm}
% l - left, r - right, c - center. | means one vertical line
\begin{tabular}{cc}
\toprule
\textbf{名称}&\textbf{描述}\\
\midrule
\textbf{DEBUG\_HANDLER}&默认机制,控制台打印堆栈信息\\ \hline
\multirow{2}*{\textbf{HTML\_DEBUG\_HANDLER}}&适合使用FreeMarker输出HTML页面时使用,\\
~&支持堆栈信息格式化\\ \hline
\textbf{IGNORE\_HANDLER}&直接忽略所有异常情况\\ \hline
\textbf{RETHROW\_HANDLER}&重新抛出异常,适合Web应用等场景\\
\bottomrule
\end{tabular}
\label{table:exception}
\end{table}

\section{Fabric智能合约代码智能提示模块实现}

插件采用自然语言处理模型GPT-2为此模块提供支持,增强IntelliJ IDE对Fabric Node.js智能合约的代码补全效果。
实现步骤主要包括:数据采集、数据预处理、模型训练、模型应用。
下面将详细展开介绍。

\subsection{数据采集}

基于需求分析,此模块需要针对Fabric Node.js智能合约的编码给出智能提示,那么Fabric Node.js智能合约即为GPT-2模型需要学习并生成的文本数据。
因此,模块选择GithHub开源网站进行数据采集,使用Python语言编写网页爬虫爬取智能合约文件作为模型输入。
网页爬虫的核心是基于URL过滤并锁定数据,从而自动爬取特征范围内的网页。爬虫代码实现片段如图~\ref{fig:5.6}所示。

\begin{figure}[htb]
  \centering
  \includegraphics[width=\linewidth]{figure/implement/爬虫实现.pdf}
  \caption{爬虫实现}\label{fig:5.6}
\end{figure}

Hyperledger Fabric为Node.js智能合约的编写提供了两种开发包,在智能合约中必须引入二者之一。
因此,模块采集的数据特征是代码文件中包含关键字“fabric-shim”或“fabric-contract-api”。
通过关键字搜索的文件结果数量远大于网站规定的可显示数量,因此在URL中采取限制文件大小的方式缩小结果数量,以此抓取所有满足条件的智能合约。
经过对智能合约和GitHub网站的深入分析,得到上图代码需要爬取的URL为“https://github.com/search?l=JavaScript\&p=num\&q=require(fabric-shim)+size\%3A \linebreak >10000\&type=Code”。
Python爬虫最终抓取的Fabric Node.js智能合约总计6151份。

\subsection{数据预处理}

数据预处理分为两个部分,一是数据清洗,二是分词编码。

数据清洗主要针对智能合约文件中的注释。
因为插件的智能提示目标是代码段,所以注释不是模型的学习对象,其存在反而会影响代码的上下文。

代码文件中的注释有两种类型,第一种是以“//”开头的单行注释,第二种是被“/*”和“*/”包裹的多行注释。
数据清洗脚本首先将代码复制到一个临时文件中,接着用指针遍历文件中的字符,识别关键的起始符号以确认是否是注释。
非注释的代码语句都将被重新写入文件中,最后,临时文件会被删除。

对于清理注释的方法,一共设置了七个状态,以便指针运动的过程中能够分辨出当前字符是否在注释中。
首先文档的状态初始设置为INIT状态。
指针遍历文档中每一行的每一个字符,读入字符时判断当前状态:

\begin{itemize}
  \item INIT状态:如果当前字符为"/",文档状态切换到SLASH,以标记遇到了第一个"/",接着遍历下一个字符;如果当前字符为引号,则表明进入了一对引号引起的字符串,此时将状态切换到STR,并将当前字符写入文件;在其他情况下,说明当前不在注释内,直接写入当前字符。
  \item SLASH状态:如果当前字符为"*",表明了一个多行块状注释的起始,此时将状态切换为BLOCK\_COMMENT;若当前字符为"/",表明了一个单行注释的起始,此时将状态切换为LINE\_COMMENT;若当前字符为其他字符,则将前一位的"/"和当前字符一起写入文件,并将状态切换回INIT。
  \item BLOCK\_COMMENT状态:如果当前字符为"*",则表明可能已遇到块状注释结尾的倒数第二个字符,此时将状态切换为BLOCK\_COMMENT\_DOT;若为其他字符,表明当前还在注释块中,直接进入下一个字符。
  \item BLOCK\_COMMENT\_DOT状态:如果当前字符为"/",表明当前注释块已到结尾,此时状态切换回INIT;若为其他字符,则表明还在注释块中,进入下一个字符。
  \item LINE\_COMMENT状态:由于该行均为注释,只有当前字符为换行符时,才将状态切换为INIT,接着进入下一行。
  \item STR状态:表明当前在字符串中。若当前字符为反斜杠,则将状态切换为STR\_ESCAPE;若当前字符为引号,表明字符串结束,状态切换为INIT;若当前字符为其他字符,状态保持不变。最后写入当前字符。
  \item STR\_ESCAPE状态:将状态切换回STR,接着写入当前字符到文件,并进入下一个字符。
\end{itemize}

模块采用的分词器(Tokenizer)使用UTF-8编码方式读取数据,为了防止Tokenizer加载数据异常,在分词编码之前会对数据的编码格式进行清洗。

Tokenizer利用字节对编码(Byte Pair Encoder,BPE)对文本序列进行分词,代码实现片段如图~\ref{fig:5.7}所示。

\begin{figure}[htb]
  \centering
  \includegraphics[width=0.8\linewidth]{figure/implement/分词编码实现.pdf}
  \caption{分词编码实现}\label{fig:5.7}
\end{figure}

BPE会根据统计数据和词汇表对单词序列进行拆分,得到不同的字母组合,单词序列经过编码后会得到多个ID,模型预测的结果ID会经过解码得到字母组合。
上图代码中~\texttt{bpe\_train()}为训练分词模型的方法,使用定义好的分词模型对Node.js智能合约文件进行分词和编码,并将相关信息保存在文件中。


\subsection{GPT-2模型训练}

由于自然语言模型的学习过程对硬件需求较高,所以选择将模型部署在GPU机器上。
使用Anaconda环境管理工具隔离模型执行所依赖的环境。
安装CUDA和TensorFlow框架支持GPT-2模型的搭建和训练。

首先编写脚本准备输入数据,代码实现片段如图~\ref{fig:5.8}所示。

\begin{figure}[htb]
  \centering
  \includegraphics[width=0.85\linewidth]{figure/implement/准备模型输入数据脚本.pdf}
  \caption{切分数据样本脚本}\label{fig:5.8}
\end{figure}

脚本将每个智能合约文件中的语句连接起来形成一个字符串,其中文件与文件之间加入特殊字符以便标识文件的结尾,Tokenizer将得到的字符串按照词汇表进行分词编码。
然后将已成为编码序列的数据切分成多个块,块中容纳100个词,对于每一个块,将数据根据上下文分为输入集和标签集。
将输入集划分为训练集、验证集和测试集,其中训练集和测试集按照8:2的比例切分,而训练集中又按照8:2的比例分割成最终的训练集和验证集。

TensorFlow会加载准备好的数据集,并根据设定的buffer size和batch size将数据打乱。

GPT-2模型底层基于Transformer解码器,采用transformers包提供的GPT2Config、TFGPT2LMHeadModel和GPT2Tokenizer包实现模型训练过程,代码实现片段如图~\ref{fig:5.9}所示。

\begin{figure}[htb]
  \centering
  \includegraphics[width=0.85\linewidth]{figure/implement/模型训练脚本.pdf}
  \caption{模型训练脚本}\label{fig:5.9}
\end{figure}

脚本首先加载已保存的分词模型,使用分词模型的词汇表作为配置注入GPT-2并形成适应数据的模型。
模型训练的优化函数使用了~\texttt{Adam()};损失函数使用的是稀疏类别交叉熵。
脚本设置了回调函数以帮助模型在性能无法提升时能够提前终止训练,回调函数监控的指标为验证集的损失(Loss)。

同时,对训练过程中训练集和验证集的Loss进行记录,并基于MatPlotLib进行可视化,便于观察和评估模型训练效果。
如图~\ref{fig:5.10}所示。

\begin{figure}[htb]
  \centering
  \includegraphics[width=0.6\linewidth]{figure/implement/Loss.png}
  \caption{训练集和验证机Loss可视化}\label{fig:5.10}
\end{figure}

数据可视化后可以清晰地观察到,模型在验证集中的Loss在第十轮训练后已经趋于平缓,且根据回调函数可知,Loss在第二十一轮开始几乎不再下降,所以认为模型训练可以终止。

\subsection{基于GPT-2模型的代码智能提示的实现}

上文介绍了代码智能提示模块的数据采集、数据预处理以及模型训练等步骤的实现细节。
训练好的GPT-2模型共计451MB,包含了模型结构、分词器、词汇表等内容,如图~\ref{fig:5.11}所示。

\begin{figure}[htb]
  \centering
  \includegraphics[width=0.4\linewidth]{figure/implement/GPT-2目录.png}
  \caption{GPT-2模型结构}\label{fig:5.11}
\end{figure}

由需求分析和系统设计可知,代码智能提示模块采用了Flask Web框架提供自然语言处理服务。
Flask框架需要进行环境配置,使用pip install命令安装flask及相关依赖包。

模块中编写的Flask应用程序依赖OpenAI官方提供的两个开发包GPT2To- \linebreak kenizer和TFGPT2LMHeadModel,调用~\texttt{from\_pretrained()}方法加载分词器和GPT-2模型。
Flask服务器提供RESTFUL接口以接收HTTP请求传送的数据,调用分词器的~\texttt{encode()}方法对输入数据进行编码,调用模型的~\texttt{generate()}方法得到模型预测结果,Flask应用程序代码实现片段如图~\ref{fig:5.12}所示。

\begin{figure}[htb]
  \centering
  \includegraphics[width=0.75\linewidth]{figure/implement/Flask应用程序实现.pdf}
  \caption{Flask应用程序实现}\label{fig:5.12}
\end{figure}

启动Flask服务器预先对Tokenizer和GPT-2模型进行加载,为插件的实时代码补全功能提供高稳定和高性能的支持,同时使得插件开发和自然语言处理服务实现良好解耦,弱化彼此牵制的影响。

此模块基于IntelliJ平台提供的CompletionContributor接口实现了代码自动补全扩展点,并在plugin.xml中注册该扩展点使其生效,配置文件代码片段如图~\ref{fig:5.13}所示。
由于此插件面向Node.js智能合约的开发,因此,设置该扩展点只在开发者编写JavaScript类型文件时生效。

\begin{figure}[htb]
  \centering
  \includegraphics[width=\linewidth]{figure/implement/代码自动补全扩展点注册实现.pdf}
  \caption{代码自动补全扩展点注册实现}\label{fig:5.13}
\end{figure}

插件在completion.contributor扩展点中制定了自定义的代码自动补全机制GPTCompletionContributor,该类需要实现CompletionContributor接口,重写~\texttt{fillCompletionVariants()}方法,下文将阐述该方法的业务逻辑。

首先,在方法中利用CompletionParameters获取当前编辑窗口键入序列,代码实现片段如图~\ref{fig:5.14}所示。

\begin{figure}[htb]
  \centering
  \includegraphics[width=0.85\linewidth]{figure/implement/获取键入文本序列实现.pdf}
  \caption{获取键入文本序列实现}\label{fig:5.14}
\end{figure}

IntelliJ平台提供了程序结构接口(Program Structure Interface,PSI),插件利用PSI元素探索了IntelliJ平台的源代码结构,执行代码分析。
如上文代码片段所示,CompletionParameters封装了多个PsiElement属性。
调用~\texttt{getOriginalPosition()}方法返回的original对象是当前光标所在代码行的PsiElement,调用~\texttt{parameters.getOffset()}获取当前代码行内输入字符的偏移量,调用~\texttt{parameters.getPosition().getTextRange().get-} \linebreak ~\texttt{StartOffset()}获取当前代码行内输入文本序列前缀的起始偏移量,两者之差textLength即为当前代码行起始下标到光标之间的长度,从original对象中截取长度为textLength的字符串得到当前编辑窗口键入的文本序列。

获取模型输入后,在~\texttt{fillCompletionVariants()}方法中调用Flask服务器的对外暴露接口发送文本序列,获取GPT-2模型预测的结果集,最后将结果集添加到代码自动补全列表中,代码实现片段如图~\ref{fig:5.15}所示。

\begin{figure}[htb]
  \centering
  \includegraphics[width=0.7\linewidth]{figure/implement/代码智能提示实现.pdf}
  \caption{代码智能提示实现}\label{fig:5.15}
\end{figure}

IntelliJ平台将代码自动补全列表中的每个条目封装成LookupElement类,使用LookupElementBuilder创建新的元素,最后调用~\texttt{addElement()}方法将元素加入CompletionResultSet中。

\section{Fabric智能合约模拟调用及测试模块实现}

Fabric智能合约模拟调用模块的底层基于FreeMarker模板技术,结合Hyperledger Fabric提供的模拟桩ChaincodeMockStub和JavaScript测试框架Mocha,实现了智能合约接口的模拟调用、调试和测试功能。
同时,插件也为智能合约开发者提供了友好的用户交互接口。

\subsection{模拟调用及测试的实现}

智能合约在实际的Hyperledger Fabric网络中的执行过程如图~\ref{fig:5.16}所示。
插件依据智能合约链上的执行原理实现了链下对智能合约接口的模拟调用。

\begin{figure}[htb]
  \centering
  \includegraphics[width=0.7\linewidth]{figure/implement/智能合约执行过程.png}
  \caption{智能合约执行过程}\label{fig:5.16}
\end{figure}

插件搭建了Node.js环境支持Mocha框架的使用,该框架作为代码质量保证的工具,对JavaScript代码的测试提供了强大的支持,功能丰富且灵活,并且可以提供准确和简洁的测试报告。

在Fabric智能合约模拟调用模块中,FreeMarker模板引擎加载的FTL模板内容是基于Mocha框架编写的智能合约接口测试脚本,脚本利用ChaincodeMockStub调用智能合约接口并获取返回值,同时使用其内置的键值对数据结构模拟Hyperledger Fabric网络的状态数据库。
FTL模板代码实现片段如图~\ref{fig:5.17}所示。

\begin{figure}[htb]
  \centering
  \includegraphics[width=\linewidth]{figure/implement/FTL模板实现.pdf}
  \caption{FTL模板实现}\label{fig:5.17}
\end{figure}

如上图所示,测试脚本中的每一个describe和it代码块都可以独立执行,describe代码块是脚本中的测试套件,在该代码块内的测试具有相关性,可以为describe测试套件设置名称和执行函数;
it代码块作为测试的最小单元,是脚本中的测试用例,与describe测试套件类似,也可以为it测试用例设置名称和执行函数。

Mocha还提供了对测试用例使用钩子的支持,如表~\ref{table:mocha}所示。
模块需要实现对Fabric智能合约接口的模拟调用,因此执行测试脚本describe测试套件的所有it测试用例之前,采用了Mocha框架的钩子之一~\texttt{before()}。
在~\texttt{before()}的执行函数中利用ChaincodeMockStub为Fabric智能合约的ChaincodeStubInterface创建模拟桩,并对智能合约进行初始化。

\begin{table}[htb]\scriptsize
\centering
\caption{Mocha钩子执行时间}
\vspace{2mm}
% l - left, r - right, c - center. | means one vertical line
\begin{tabular}{cc}
\toprule
\textbf{函数}&\textbf{执行时间}\\
\midrule
\textbf{before()}&在describe测试套件内的所有it测试用例之前执行\\ \hline
\textbf{after()}&在describe测试套件内的所有it测试用例之后执行\\ \hline
\textbf{beforeEach()}&在describe测试套件内的每个it测试用例之前执行\\ \hline
\textbf{afterEach()}&在describe测试套件内的每个it测试用例之后执行\\
\bottomrule
\end{tabular}
\label{table:mocha}
\end{table}

插件将智能合约名称、接口封装成数据模型融入FTL模板中,FreeMarker引擎加载模板输出一份智能合约模拟调用及测试脚本。
插件利用Mocha框架执行脚本并获取命令行打印结果,命令行输出分为标准输出和错误输出,为了防止两者之间互相阻塞,模块采用多线程的形式获取脚本执行日志,代码实现片段如图~\ref{fig:5.18}所示。
插件通过设置不同的Mocha命令行执行参数配置脚本的执行方式、控制测试报告的生成形式,执行参数的设置可以写入配置文件。
Mocha框架提供的测试报告中包含了脚本运行花费的时间,执行测试用例成功和失败的个数及其具体执行细节。

\begin{figure}[htb]
  \centering
  \includegraphics[width=0.8\linewidth]{figure/implement/执行脚本实现.pdf}
  \caption{执行脚本实现}\label{fig:5.18}
\end{figure}

\subsection{IntelliJ IDE工具窗口的实现}

Fabric智能合约模拟调用模块为了提高功能易用性,基于IntelliJ平台提供的用户接口组件Tool Windows实现了友好的用户接口层。

模块定义了MockerToolWindowFactory,该类作为工具窗口的工厂实现了ToolWindowFactory接口,在plugin.xml配置文件中注册扩展点toolWindow使得自定义的工具窗口在IntelliJ IDE界面生效,注册工具窗口代码实现片段如图~\ref{fig:5.19}所示。

\begin{figure}[htb]
  \centering
  \includegraphics[width=\linewidth]{figure/implement/注册工具窗口.pdf}
  \caption{注册工具窗口}\label{fig:5.19}
\end{figure}

配置文件中自定义工具窗口的属性id标识了窗口名称。IntelliJ IDE中的左侧、底部及右侧的工具窗口区域都包含主要组和次要组,其中次要组中的工具窗口不能同时打开,配置文件中属性secondary设为true,表示该工具窗口是次要的。
配置文件中的属性anchor设置为right,则该工具窗口在界面右侧区域展开。
属性factoryClass指定了自定义工具窗口的实现工厂。

当智能合约开发者打开工具窗口时,MockerToolWindowFactory重写接口的~\texttt{createToolWindowContent()}方法将被调用,初始化工具窗口UI,该机制保证了工具窗口未打开时并不会占用内存而影响IDE性能。
MockerToolWindowFactory工厂类的代码实现片段如图~\ref{fig:5.20}所示。
方法中调用了ContentFactory的~\texttt{createContent()}方法为工具窗口创建组件内容,然后获取到工具窗口的ContentManager并将内容添加到工具窗口。

\begin{figure}[htb]
  \centering
  \includegraphics[width=\linewidth]{figure/implement/自定义工具窗口工厂类实现.pdf}
  \caption{自定义工具窗口工厂类实现}\label{fig:5.20}
\end{figure}

MockerToolWindow基于Java Swing的开发包实现了智能合约模拟调用工具窗口的GUI。
该窗口由Tool Bar、智能合约接口列表面板、自定义控制台三部分构成。
Tool Bar提供了带有监听器的按钮,开发者点击相应按钮可以查询智能合约及其接口、模拟调用、重置面板等。
插件实现了自定义的TableCellRender渲染器,以此控制智能合约接口列表的格式。
自定义控制台对执行智能合约模拟调用脚本的输出日志和结果进行展示。
工具窗口界面如图~\ref{fig:5.21}所示。

\begin{figure}[htb]
  \centering
  \includegraphics[width=\linewidth]{figure/implement/工具窗口.png}
  \caption{智能合约模拟调用工具窗口}\label{fig:5.21}
\end{figure}

\section{本章小结}

本章基于插件的概要设计和详细设计对各个模块功能进行代码实现和结果展示。结合插件的关键代码分别阐述了Fabric智能合约初始模板生成、代码智能提示、模拟调用三个模块的开发细节。
