\chapter{绪论}

\section{项目背景}

互联网的诞生便于人们在金融、物联网、供应链、医疗等多种领域进行数据共享。
这些数据共享的场景通常是由多个组织或个人共同参与的,由于多方之间存在信任危机,因此互联网对于他们而言是存在安全隐患的。
自2008年中本聪首次提出区块链的概念以来~\cite{Bitcoin},区块链技术就为涉及利益交互的各种应用领域提供了一种解决信任问题的方式,
它可以让各个组织或个人制定自己的交易规则,并且所有的交易记录都是可以追踪溯源的。

区块链的各个节点持有完全一致的数据,交易各方之间不存在一个可以进行干涉的中心节点。
这种去中心化的架构配合底层的密码学技术~\cite{2016区块链技术发展现状与展望},使得数据无法篡改,保证了交易的安全性,如图~\ref{fig:1.1}所示。
区块链成为了存储数据的保险箱,这就是它最大的价值所在。

\begin{figure}[htb]
  \centering
  \includegraphics[width=0.85\linewidth]{figure/intro/区块链网络结构.png}
  \caption{区块链网络结构}\label{fig:1.1}
\end{figure}

对于区块链领域的开发工作而言,智能合约起到了关键性的作用,因为它是开发人员与区块链上的账本数据进行交互的唯一途径。
智能合约这一概念最早是由一名跨多领域的法律学者Nick Szabo于20世纪90年代提出的~\cite{王春宇2016智能合约与金融合约}。
随着以太坊为代表的面向公有链的开源平台不断出现,智能合约的执行有了可信环境~\cite{QianYang},开始应用在区块链领域。

智能合约和日常生活中的各种合约相似,都是基于各个参与者协商后的结果来制定一系列规则,以此约束某一场景下的行为。
智能合约作为业务逻辑的承载者,是运行在区块链分布式节点上的一段程序。
在区块链场景下,各个组织或者个人需要按照程序中规定的业务逻辑执行交易。
如今区块链技术不仅仅应用在金融领域,而是扩展到日常生活的方方面面,融入到传统行业中提供数字化转型的解决方案。

由Linux基金会主导发起的Hyperledger Fabric~\cite{Dhillon}作为面向联盟链的区块链开源平台,成为了企业级应用的首选。
Hyperledger Fabric这一架构利用独特的共识协议,使得交易的处理速度不再像以太坊的公有链那样受到较大的限制,由每分钟几百笔交易提升到了每分钟大约50万笔交易~\cite{DBLP:journals/access/PinnaIBTM19}。
智能合约的开发也得到了多种编程语言的支持,例如当前主流的开发语言Java、Go、Node.js等,为企业的开发人员提供了更多选择。

然而,区块链开发人员对于Hyperledger Fabric智能合约的编写主要还停留在使用文本编辑器的阶段,开发环境较为简陋,缺少IDE能够提供的代码提示功能,从而导致了智能合约开发效率低下~\cite{罗雄2020}。
这意味着开发人员的编码工作需要耗费更多时间,而且代码质量从基本的语法和上下文语义的角度都无法得到保证。
将这样漏洞频发的智能合约代码部署到区块链节点上会引发一系列灾难。

首先,在实际开发环境中,当智能合约在区块链网络节点上运行报错时,开发人员需要进行调试工作。
然而,当前主流的IDE没有提供针对智能合约调试或者测试相关的功能,使用简单的文本编辑器来修改出错的智能合约将耗费大量的时间和精力。

其次,区块链的优势之一就是它的不可篡改性,这为涉及利益交互的交易参与者提供了良好的安全保障,
但是同时,这一特点意味着对智能合约的每一次修改调试,都可能需要重新启动区块链网络,创建通道并且将节点加入通道,智能合约还需要经历安装、实例化等生命周期中的必要操作才能在区块链节点上运行。

再次,区块链实际应用场景和开发者部署一个单节点的测试网络不同,需要大量网络服务器节点,而每次对有漏洞的智能合约做出改动后都需要在每一个节点上进行上述操作~\cite{XiaozhouYang},这是一项冗余繁杂的工作。

即使开发人员编写了一份完美的智能合约,它能够在区块链网络节点上被SDK、CLI等客户端调用仍然需要具备一个前提——搭建并部署一个可以正常工作的区块链网络,而这些繁杂的步骤对于智能合约开发工作而言却不是一个必要条件。

因此,需要一个针对Hyperledger Fabric智能合约开发的辅助工具,能够方便开发人员在链下进行智能合约的开发,在将智能合约发布到区块链网络节点之前就确保其实现和业务逻辑的正确性,从而提升代码质量和部署效率。
本文基于以上项目背景研究解决方案,实现了一种面向Hyperledger Fabric的智能合约开发插件。

\section{国内外相关研究和应用现状}

区块链作为一个可溯源、不可篡改、去中心化的分布式账本,提供了一种有价值的机制可以让多方安全地共享数据并且降低成本。区块链技术从概念的提出到如今的广泛应用,其发展大致分为以下三个阶段。

“区块链1.0”阶段聚焦在可编程货币上,中本聪先生提出的比特币为互联网实现货币交易提供了可能性,越来越多的虚拟数字货币因为区块链技术而诞生。区块链开始在金融领域创建新的电子货币体系~\cite{DBLP:journals/hmd/Portmann18b},使得世界各地的人们都可以依靠互联网进行安全且便捷的买卖交易。

“区块链2.0”阶段引入了智能合约的概念~\cite{Sheikh},出现了以以太坊(Ethere- ~\linebreak um)为代表的区块链平台,区块链开始向股票、期权等金融衍生领域扩展~\cite{2015Blockchain}。以太坊通过智能合约让参与交易的组织或个人可以制定一份需要彼此严格遵守的协议,在这个交易过程中不再需要第三方中介。

“区块链3.0”阶段开始向日常生活的各行各业发展,只要是涉及信任问题的场景,都有区块链发挥价值的可能性。不同于面向公有链的以太坊,超级账本Hyperledger Fabric的出现为企业级的联盟链提供了平台,具有更高的商业价值。同时,交易效率也得到了极大提升。

如今区块链技术的价值愈加体现,国内外越来越多的专家学者开始涉入这一新兴领域进行学术研究~\cite{张荣2017, 陈旭2017},也有越来越多的企业将这一技术落地到实际项目中,而智能合约在区块链架构中扮演着不可或缺的角色,其开发也逐渐受到重视。

以太坊平台支持使用图灵完备语言进行智能合约编写,例如类似Java的高级编程语言Solidity、类似Python的脚本语言Serpent或者类似Assembly的低级语言LLL。
其中Solidity作为一种专门为以太坊智能合约设计的面向对象的高级编程语言,受到了C++、JavaScript等语言的启发,使用最为广泛,可以运行在以太坊虚拟机(Ethereum Virtual Machine,EVM)上。

区块链的实际应用场景可能涉及复杂的背景和用户需求,那么智能合约所承载的业务逻辑的体量也会随之增大,这样具有一定规模的智能合约程序代码已经不适合使用简单的文本编辑器进行开发,最好的方式就是使用一个能够提供编辑、编译、调试等功能的集成开发环境。

随着以太坊的发展,相关开源社区提供了一些智能合约的集成开发环境,例如Remix、Truffle等~\cite{QianYang},其中Remix作为以太坊官方推荐工具尤为主流。
Remix IDE是一款支持开发者在线使用浏览器进行智能合约开发的Web端工具,它针对Solidity语言编写的智能合约提供了代码编辑、编译、调试、测试等功能,这些功能模块都是以插件的形式存在的。
Remix IDE界面如图~\ref{fig:1.2}所示,由图标面板、侧方插件面板、主开发面板以及终端四部分组成。

\begin{figure}[htb]
  \centering
  \includegraphics[width=\linewidth]{figure/intro/Remix界面.png}
  \caption{Remix IDE开发界面}\label{fig:1.2}
\end{figure}

图标面板是Remix IDE支持的各个功能插件的入口,选中后图标对应的插件会出现在侧方插件面板中;侧方插件面板主要提供了插件独立的图形用户界面(Graphical User Interface,GUI),使得开发者完成相关配置内容或者填写具体参数;主开发面板是编写Solidity智能合约的区域,与传统IDE类似,在开发者编码的同时,给出代码补全、代码高亮、代码错误提示等内容;终端使得开发者能够和开发工具GUI进行交互,并且支持脚本的运行。

进入区块链3.0阶段后,企业操作系统(Enterprise Operating System,EOS)区块链平台成为了有力的竞争者之一。由于EOS智能合约的开发语言是C++,所以开发人员通常使用VScode IDE 作为EOS智能合约的开发工具。EOSFactory也提供了VScode IDE的EOS智能合约开发插件~\cite{罗雄2020},但其并没有针对运行、调试、测试方面的支持。

Hyperledger Fabric自2018年作为开源项目发布以来,成为了针对企业级联盟链的区块链架构代表。由于引入了身份认证的角色,各个组织或者个人可以规定数据公开范围,对联盟的创建提供了友好的机制,并且保证了数据的私密性和安全性。
相比以太坊,Hyperledger Fabric在智能合约上也做了很多改进和扩展工作,使得开发人员可以使用Java、Go、Node.js等多种当前主流的开发语言进行智能合约编码。
Hyperledger Fabric具有可配置、模块化、性能高等众多优势,但是该技术较新且仍然在发展阶段。
因此,选择Hyperledger Fabric作为本文研究对象是具有一定必要性的。

目前,Hyperledger Fabric已经从最初的1.x版本升级到了2.x版本,新版本使其具有更多的优势。加强了对于智能合约的去中心化管理;针对联盟链内的合作和共识提供了新的链码应用模式;添加了新功能使得数据私密性得到了更强的保障等。从智能合约的编码角度考虑,ACE作为一款高性能的Web编辑器,可以支持120多种语言的语法高亮~\cite{罗雄2020},ACE代码编辑器基于脚本语言JavaScript开发,支持与JavaScript应用或者Web页面结合使用。然而,在Hyperledger Fabric智能合约的开发上却缺少一个针对性的集成开发环境,用以提供编辑、运行、调试、测试等重要功能,因此区块链开发者仍然面临开发效率、代码质量带来的挑战。

\section{解决方案及创新}

由于区块链网络的搭建和部署工作较为复杂,而且针对智能合约的开发而言,区块链网络及其相关内容并不是必要条件,因此本文研究并提出了一种无需区块链网络的链下智能合约开发工具。

同时,考虑到可扩展性和易用性,该工具作为一个基于IntelliJ平台的IDE插件,提供轻量级的开发辅助支持。为了让开发者在区块链网络上发布智能合约之前就可以保证其正确性,该插件需要实现以下几点功能:

\begin{itemize}
    \item 智能合约初始模板生成:对于智能合约的编写,Hyperledger Fabric 1.x版本提供了基于fabric-shim包的开发方式,而升级后的2.x版本增添了新的基于fabric-contract-api的开发方式,插件能够让用户根据不同场景下的需求,选择创建合适的初始智能合约模板,提高开发效率。
    \item 智能合约代码智能提示:在程序代码编写过程中,给出实时的智能代码补全提示,降低开发门槛并且提升编码速度。
    \item 智能合约模拟调用与测试:开发人员完成智能合约编写工作后,可以进行接口的模拟调用,获取调用结果。
\end{itemize}

\section{主要工作}

本文以软件工程的思想作为根本,经过需求分析、系统整体设计和模块详细设计,并基于以上给出了解决方案的具体实现和测试。
本文指出了当前区块链领域中智能合约开发存在的问题并完成了面向Hyperledger Fabric的智能合约开发插件的设计与实现:

\begin{itemize}
    \item 首先,本文详细地介绍了开发这款插件的项目背景。通过对区块链领域研究现状的描述与分析,引入了目前区块链开发人员在Hyperledger Fabric智能合约开发过程中的问题和难点。阐述了提供一种链下辅助工具的意义、重要程度及其优势。
    \item 本文简洁地介绍了开发这款插件所涉及的关键技术,包括区块链领域的相关技术、集成开发环境中插件的相关技术以及自然语言处理的相关技术。
    \item 本文深入地分析了面向Hyperledger Fabric智能合约的开发插件的功能性需求和非功能性需求,将插件切分成不同的模块进行需求的描述和分析,并基于需求分析进行项目的整体架构设计,由此对插件各个子模块进行详细的设计。
    \item 本文根据项目的需求分析和系统设计进行了具体的代码实现,给出了对于创建初始智能合约、智能合约代码提示、智能合约模拟调用等重要功能的关键代码。
    \item 最后,本文展示了对于插件各个功能的测试情况,包括测试所用的技术和具体测试流程,并给出了测试结果。本文结合测试结果总结了所述解决方案的有效性和不足,提出了后续优化思路。
\end{itemize}

\section{组织结构}

本文的组织结构如下:

第一章,绪论部分。
介绍了项目的背景和相关应用发展现状,阐明面向Hyperledger Fabric的智能合约开发插件所能解决的问题和解决方案,并简要列举了本文的工作内容。

第二章,技术综述。
对这一智能合约辅助工具的开发所涉及的关键技术给出了简洁的介绍,包括区块链、Hyperledger Fabric、IntelliJ插件、自然语言处理及其相关技术等,并且解释了选择该技术解决问题的理由。

第三章,插件的需求分析与设计。
对插件的功能需求和非功能性需求进行了细致分析。
简洁阐明了插件总体架构以及各个模块的划分思路,并且详细描述了插件各个模块的设计方案。

第四章,插件的实现。
基于第三章的需求分析和设计,描述了智能合约开发插件的实现细节,展示了插件核心功能的关键代码及其插件运行的界面图。

第五章,插件的测试。
对插件的功能进行测试并对整体运行进行性能测试,展示并分析测试结果。

第六章,总结与展望。对本论文的工作和成果做出总结,并且提出了智能合约开发插件的可优化思路,对插件的可扩展点做了进一步的展望。
