\chapter{插件的测试}

本章主要介绍面向Hyperledger Fabric的智能合约开发插件的测试工作。首先概述了测试的目标及方式;然后基于模块的功能划分设计了测试用例从而完成功能测试,同时对插件应用的自然语言处理模型做了评估;
最后,基于插件的非功能性需求进行了性能测试。

\section{测试目标及方式}

本文始终依照软件工程方法执行每一个环节,在插件的整个研发过程中,测试工作的跟进具有重要价值。

本章的测试目标主要有以下几点:

\begin{itemize}
    \item 验证插件各模块的功能实现是否满足智能合约开发者的功能性需求。
    \item 保证插件所提供服务具有稳定性且满足性能要求。
    \item 评估插件的质量,为插件的开发提供依据和帮助。
    \item 检查插件的开发是否存在漏洞,并发现插件值得改进和扩展的地方。
\end{itemize}

在确定测试目标后,围绕功能测试和性能测试展开测试工作。由于插件基于IntelliJ平台开发,因此对插件的测试基于IntelliJ平台的测试框架,通过扩展LightCodeInsightFixtureTestCase类实现了插件的轻量级测试。

插件的测试环境分为两部分,一个是本地开发环境,一个是提供GPT-2自然语言处理模型服务的Google Cloud环境。
两个环境及其配置信息如表~\ref{table:localEnvironment}和~\ref{table:cloudEnvironment}所示。

\begin{table}[htb]\scriptsize
\centering
\caption{本地环境配置}
\vspace{2mm}
% l - left, r - right, c - center. | means one vertical line
\begin{tabular}{ccc}
\toprule
\textbf{配置项}&\textbf{备注}\\
\midrule
\textbf{CPU}&8核\\ \hline
\textbf{内存}&8G\\ \hline
\textbf{SSD}&512G\\ \hline
\textbf{操作系统}&Windows\\ \hline
\textbf{语言环境支持}&Java、Node.js、Python\\
\bottomrule
\end{tabular}
\label{table:localEnvironment}
\end{table}

\begin{table}[htb]\scriptsize
\centering
\caption{Google Cloud环境配置}
\vspace{2mm}
% l - left, r - right, c - center. | means one vertical line
\begin{tabular}{ccc}
\toprule
\textbf{配置项}&\textbf{备注}\\
\midrule
\textbf{GPU}&NVIDIA Tesla P4\\ \hline
\textbf{内存}&15G\\ \hline
\textbf{SSD}&50G\\ \hline
\textbf{操作系统}&Ubuntu\\ \hline
\textbf{语言环境支持}&Python\\ \hline
\textbf{模型环境支持}&TensorFlow\\
\bottomrule
\end{tabular}
\label{table:cloudEnvironment}
\end{table}

\section{功能测试}

根据功能性需求分析,本章对Fabric智能合约初始模板生成模块、代码智能提示模块、模拟调用及测试模块的功能设计了测试用例,验证各功能是否能在给定输入的情况下按照预期返回结果。同时,各功能的测试代码也为重用插件代码的开发者提供了接口调用的示例。

通过测试用例的设计和测试代码的实现顺利完成了插件各模块的功能测试,测试用例全部通过,测试结果均符合预期,验证了插件各功能实现的正确性,插件可以正常运行且满足用户功能性需求,具有可用性。

\subsection{Fabric智能合约初始模板生成模块测试}

Fabric智能合约初始模板生成模块主要测试能否根据智能合约开发者需求生成符合预期的初始智能合约文件,其测试用例描述如表~\ref{table:tc1}所示。

\begin{table}[htb]\scriptsize
\centering
\caption{初始模板生成测试用例}
\vspace{2mm}
% l - left, r - right, c - center. | means one vertical line
\begin{tabular}{ccc}
\toprule
\textbf{ID}&TC1\\
\midrule
\textbf{名称}&初始模板生成\\ \hline
\textbf{测试目标}&能否根据用户选择成功创建初始智能合约\\ \hline
\textbf{前置条件}&用户当前工作空间存在项目结构\\ \hline
\textbf{输入}&智能合约名称和开发模式\\ \hline
\multirow{3}*{\textbf{测试步骤}}&1.用户输入所要创建智能合约的名称\\
~&2.用户选择智能合约开发模式\\ 
~&3.创建初始智能合约\\ \hline
\multirow{3}*{\textbf{预期结果}}&1.弹出智能合约创建成功提示框,包含路径等信息\\
~&2.生成智能合约文件\\ 
~&3.智能合约内容与相应FTL模板匹配\\
\bottomrule
\end{tabular}
\label{table:tc1}
\end{table}

\subsection{Fabric智能合约代码提示模块测试}

Fabric智能合约代码提示模块主要测试用户在IntelliJ IDE编辑框键入文本时是否具有代码自动补全效果,其测试用例描述如表~\ref{table:tc2}所示。

\begin{table}[htb]\scriptsize
\centering
\caption{代码智能提示测试用例}
\vspace{2mm}
% l - left, r - right, c - center. | means one vertical line
\begin{tabular}{ccc}
\toprule
\textbf{ID}&TC2\\
\midrule
\textbf{名称}&智能合约代码智能提示\\ \hline
\textbf{测试目标}&用户键入文本序列时是否提供代码自动补全选项\\ \hline
\textbf{前置条件}&用户编辑JavaScript文件\\ \hline
\textbf{输入}&IntelliJ IDE编辑框当前键入文本序列\\ \hline
\multirow{2}*{\textbf{测试步骤}}&1.用户键入智能合约代码\\
~&2.用户选中代码自动补全列表的备选项\\ \hline
\multirow{3}*{\textbf{预期结果}}&1.弹出代码自动补全下拉列表\\
~&2.列举模型预测结果集\\ 
~&3.补全用户键入代码序列\\
\bottomrule
\end{tabular}
\label{table:tc2}
\end{table}

IntelliJ平台的测试框架对插件功能的测试提供了很多辅助方法。
在测试代码中,可以通过调用~\texttt{type()}模拟在内存编辑器内键入字符;
通过调用~\texttt{complete()}模拟对代码自动补全列表的请求动作;
通过调用~\texttt{checkRes-} \linebreak ~\texttt{ultByFile()}方法比较预期和实际的操作结果。

插件在IntelliJ IDE中提供实时代码提示的效果很大程度上取决于GPT-2模型训练的效果。
在对GPT-2模型的训练过程中记录了逻辑准确率作为重要的评估指标之一,并将历史数据可视化便于观察模型训练结果,如图~\ref{fig:6.1}所示。

\begin{figure}[htb]
  \centering
  \includegraphics[width=0.6\linewidth]{figure/implement/Accuracy.png}
  \caption{训练集和验证集的逻辑准确率}\label{fig:6.1}
\end{figure}

如上图所示,GPT-2模型训练集和验证集的逻辑准确率在训练第十轮左右趋于平缓,训练停止时验证集的准确率已接近95\%。
由此可见,经过数据采集、数据预处理以及模型训练与评估的反复迭代过程,GPT-2模型的精度得到了改善。

\subsection{Fabric智能合约模拟调用模块测试}

Fabric智能合约模拟调用模块主要测试用户能否调用和测试智能合约接口,其测试用例描述如表~\ref{table:tc3}所示。

\begin{table}[htb]\scriptsize
\centering
\caption{模拟调用测试用例}
\vspace{2mm}
% l - left, r - right, c - center. | means one vertical line
\begin{tabular}{ccc}
\toprule
\textbf{ID}&TC3\\
\midrule
\textbf{名称}&智能合约模拟调用\\ \hline
\textbf{测试目标}&用户能否成功获取并调用智能合约接口\\ \hline
\textbf{前置条件}&存在相应智能合约\\ \hline
\textbf{输入}&智能合约名称、接口名称及参数\\ \hline
\multirow{4}*{\textbf{测试步骤}}&1.打开模拟调用工具窗口\\
~&2.点击按钮获取智能合约接口列表\\ 
~&3.填写智能合约接口对应参数\\
~&4.点击按钮调用智能合约接口\\ \hline
\multirow{2}*{\textbf{预期结果}}&1.显示智能合约接口列表\\
~&2.在工具窗口的自定义控制台显示脚本执行报告\\ 
\bottomrule
\end{tabular}
\label{table:tc3}
\end{table}

\section{性能测试}

性能测试部分主要的目标对象是插件的代码智能提示模块和模拟调用模块。插件可以运行在基于IntelliJ平台开发的IDE中,测试显示插件满足各项性能指标,不会影响IDE其他功能的提供。
表~\ref{table:test}展示了插件各项性能指标及其测试结果。

\begin{table}[htb]\scriptsize
\centering
\caption{性能测试}
\vspace{2mm}
% l - left, r - right, c - center. | means one vertical line
\begin{tabular}{ccc}
\toprule
\textbf{对象}&\textbf{指标}&\textbf{测试结果}\\
\midrule
\textbf{代码智能提示}&可用性&>=99.999\%\\ \hline
\textbf{代码智能提示}&HTTP请求延迟&<1ms\\ \hline
\textbf{插件}&CPU占用率&max<60\%\\ \hline
\textbf{模拟调用}&平均响应时间&<10ms\\
\bottomrule
\end{tabular}
\label{table:test}
\end{table}

\section{本章小结}

本章基于第三章功能性和非功能性需求的分析,对Fabric智能合约初始模板生成、代码智能提示、模拟调用及测试模块的功能进行测试工作。
经过功能测试和性能测试,确保了插件功能的可用性和稳定性,同时插件在IntelliJ IDE中运行的性能满足一定要求。