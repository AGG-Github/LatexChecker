% !Mode:: "TeX:UTF-8"
%# -*- coding:utf-8 -*-

%% 南京大学学位论文的示例文档
%% 作者:njuhan: https://github.com/njuHan
%% 源模版repo: https://github.com/njuHan/njuthesis-nju-thesis-template

\documentclass[winfonts,master,twoside]{njuthesis}
%% 审阅模式很重要,命令是下方使用的/blind
%% njuthesis 文档类的可选参数有:
%%   winfonts, linuxfonts, macfonts, adobefonts winfonts 选项使得文档使用Windows 系统提供的字体;linuxfonts 选项使得文档使用Linux 系统提供的字体;macfonts 选项使得文档使用Mac 系统提供的字体;adobefonts 选项使得文档使用Adobe提供的OTF中文字体(需自行下载安转)
%%   phd/master/bachelor 选择博士/硕士/学士论文
%%   twoside 或 oneside 指定排版的文档为双面打印或单面打印格式(twoside会使得chapter 章节从奇数页开始,即纸张的正面开始,因此会出现一些空白的页面)
%%   nobackinfo 取消封二页导师签名信息。注意,按照南大的规定,是需要签名页的。



%%%%%%%%%%%%%%%%%%%%%%%%%%%%%%%%%%%%%%%%%%%%%%%%%%%%%%%%%%%%%%%%%%%%%%%%%%%%%%%
% set up labelformat and labelsep for subfigure 详见: http://www.latexstudio.net/archives/8652.html
\captionsetup[subfigure]{labelformat=simple, labelsep=space}

%%%%%%%%%%%%%%%%%%%%%%%%%%%%%%%%%%%%%%%%%%%%%%%%%%%%%%%%%%%%%%%%%%%%%%%%%%%%%%%
% 设置《国家图书馆封面》的内容,仅博士论文才需要填写

% 设置论文按照《中国图书资料分类法》的分类编号
%\classification{0175.2}
% 设置论文按照《国际十进分类法UDC》的分类编号
% 该编号可在下述网址查询:http://www.udcc.org/udcsummary/php/index.php?lang=chi
%\udc{004.72}
% 国家图书馆封面上的论文标题第一行,不可换行。此属性可选,默认值为通过\title设置的标题。
%\nlctitlea{论文标题第一行}
% 国家图书馆封面上的论文标题第二行,不可换行。此属性可选,默认值为空白。
%\nlctitleb{论文标题第二行}
% 国家图书馆封面上的论文标题第三行,不可换行。此属性可选,默认值为空白。
%\nlctitlec{}
% 导师的单位名称及地址
%\supervisorinfo{南京大学计算机科学与技术系~~南京市汉口路22号~~210093}
% 答辩委员会主席
%\chairman{张三丰~~教授}
% 第一位评阅人
%\reviewera{阳顶天~~教授}
% 第二位评阅人
%\reviewerb{张无忌~~副教授}
% 第三位评阅人
%\reviewerc{黄裳~~教授}
% 第四位评阅人
%\reviewerd{郭靖~~研究员}


%%%%%%%%%%%%%%%%%%%%%%%%%%%%%%%%%%%%%%%%%%%%%%%%%%%%%%%%%%%%%%%%%%%%%%%%%%%%%%%
% 设置论文的中文封面

% 单行论文标题,不可换行
\title{}
\titlea{面向Hyperledger Fabric的}
\titleb{智能合约开发插件的设计与实现}

% 如果论文标题过长,可以分两行,第一行用\titlea{}定义,第二行用\titleb{}定义,
% 使用以下3行:
%\title{} %用于覆盖单行标题内容为空
%\titlea{长标题第一行}  %第一行标题写这里
%\titleb{长标题第二行用于长标题换行} %第二行标题写这里
% 注意: \title 不能都注释,它用于控制标题选择双行还是单行。\title{}如果内容为空,则编译\titlea{},titleb{}双行标题,否则编译单行标题


% 论文作者姓名
\author{李天歌}
% 论文作者联系电话
\telphone{18851087701}
% 论文作者电子邮件地址
\email{13889384135@163.com}
% 论文作者学生证号
\studentnum{MF1932095}
% 论文作者入学年份(年级)
\grade{2019}
% 论文作者毕业年份(届), 出版授权书的学位年度
\graduateyear{2021}
% 导师姓名职称
\supervisor{邵栋副教授,张贺教授}
% 导师的联系电话
\supervisortelphone{}
% 论文作者的学科与专业方向
\major{工程硕士(软件工程领域)}
% 论文作者的研究方向
\researchfield{软件工程}
% 论文作者所在院系的中文名称
\department{软件学院}
% 论文作者所在学校或机构的名称。此属性可选,默认值为``南京大学''。
\institute{南京大学}
% 论文的提交日期,需设置年、月、日。
\submitdate{2021年 3 月 13 日}
% 论文的答辩日期,需设置年、月、日。
\defenddate{2021年 4 月 30 日}
% 论文的定稿日期,需设置年、月、日。
% 此属性可选,若注释\date{},则默认值为最后一次编译时的日期,精确到日。
% \date{2019年5月20日}

%%%%%%%%%%%%%%%%%%%%%%%%%%%%%%%%%%%%%%%%%%%%%%%%%%%%%%%%%%%%%%%%%%%%%%%%%%%%%%%
% 设置论文的英文封面

% 论文的英文标题,不可换行
\englishtitle{The Design and Implementation of Plugin for Developing Hyperledger Fabric Smart Contracts}
% 论文作者姓名的拼音
\englishauthor{Tiange Li}
% 导师姓名职称的英文
\englishsupervisor{ Associate Professor Dong Shao, Professor He Zhang}
% 论文作者学科与专业的英文名
\englishmajor{Software Engineering}
% 论文作者所在院系的英文名称
\englishdepartment{Software Institute}
% 论文作者所在学校或机构的英文名称。此属性可选,默认值为``Nanjing University''。
\englishinstitute{Nanjing University}
% 论文完成日期的英文形式,它将出现在英文封面下方。需设置年、月、日。日期格式使用美国的日期
% 格式,即``Month day, year'',其中``Month''为月份的英文名全称,首字母大写;``day''为
% 该月中日期的阿拉伯数字表示;``year''为年份的四位阿拉伯数字表示。
% 此属性可选,若注释掉\englishdate{},则默认值为最后一次编译时的日期。
% \englishdate{May 20, 2019}

%%%%%%%%%%%%%%%%%%%%%%%%%%%%%%%%%%%%%%%%%%%%%%%%%%%%%%%%%%%%%%%%%%%%%%%%%%%%%%%
% 设置论文的中文摘要

% 设置中文摘要页面的论文标题及副标题的第一行。
% 此属性可选,其默认值为使用|\title|命令所设置的论文标题
\abstracttitlea{面向Hyperledger Fabric的}
% 设置中文摘要页面的论文标题及副标题的第二行。
% 此属性可选,其默认值为空白
\abstracttitleb{智能合约开发插件的设计与实现}

%%%%%%%%%%%%%%%%%%%%%%%%%%%%%%%%%%%%%%%%%%%%%%%%%%%%%%%%%%%%%%%%%%%%%%%%%%%%%%%
% 设置论文的英文摘要

% 设置英文摘要页面的论文标题及副标题的第一行。
% 此属性可选,其默认值为使用|\englishtitle|命令所设置的论文标题
\englishabstracttitlea{The Design and Implementation of Plugin}
% 设置英文摘要页面的论文标题及副标题的第二行。
% 此属性可选,其默认值为空白
\englishabstracttitleb{for Developing Hyperledger Fabric Smart Contracts}

%%%%%%%%%%%%%%%%%%%%%%%%%%%%%%%%%%%%%%%%%%%%%%%%%%%%%%%%%%%%%%%%%%%%%%%%%%%%%%
%% 盲审命令,空白字段设置请看 .cls文件 \newcommand*{\blind}
%% 此外,请按照盲审要求自行去掉个人简历、致谢等页面中的个人信息
%%**********************!!!非常重要的盲审命令,送审前必选!!!****************
\blind
%%**********************!!!非常重要的盲审命令,送审前必选!!!****************

%%%%%%%%%%%%%%%%%%%%%%%%%%%%%%%%%%%%%%%%%%%%%%%%%%%%%%%%%%%%%%%%%%%%%%%%%%%%%%%
\begin{document}

%%%%%%%%%%%%%%%%%%%%%%%%%%%%%%%%%%%%%%%%%%%%%%%%%%%%%%%%%%%%%%%%%%%%%%%%%%%%%%%

% 制作国家图书馆封面(博士学位论文才需要)
%\makenlctitle
% 制作中文封面
\maketitle
% 制作英文封面
\makeenglishtitle


%%%%%%%%%%%%%%%%%%%%%%%%%%%%%%%%%%%%%%%%%%%%%%%%%%%%%%%%%%%%%%%%%%%%%%%%%%%%%%%
% 开始前言部分
\frontmatter

\begin{abstract}
	近年来随着数字化转型不断加速,各行业都开始重视数据存储的方式。
	而区块链技术作为一种可溯源、不可篡改且去中心化的分布式数据存储架构,在全球掀起了热潮。
	区块链如今发展到3.0阶段,以Hyperledger Fabric为代表的区块链架构已经不仅仅能够为金融领域服务,还为监管科技、支付、物流等多种行业提供了一种降低信任成本和交易成本的新途径。
	与此同时,智能合约作为开发人员与区块链网络交互的唯一中介,其开发方式需要得到更多的关注。
	然而,当前主流的集成开发环境(Integrated Development Environment,IDE)缺少对智能合约开发的支持,一方面导致其编写难以上手,影响开发效率和代码质量;
	另一方面,智能合约在区块链网络中调试困难,每次修改都需要重新经历复杂的部署流程,而链下缺少智能合约调用和测试的工具。
	
	因此,本文针对Hyperledger Fabric Node.js智能合约设计了便于其编写、调试和测试的方案,并以IDE插件的形式实现了一种帮助开发人员在链下进行智能合约开发的工具。
	插件基于IntelliJ平台和DevKit SDK开发;
	采用FreeMarker模板技术实现初始智能合约的生成;
	应用GPT-2自然语言处理模型针对Hyperledger Fabric智能合约提供代码智能提示;
	基于Hyperledger Fabric的模拟桩、Mocha框架等实现智能合约模拟调用和测试;
	结合IDE的调试模式,支持在智能合约调用异常时添加断点进行单步调试。
	
	本文描述了一个简单易用的智能合约开发插件的设计与实现。
	该插件能够为智能合约开发者提供一个轻量级的链下开发工具,降低发布智能合约的时间损耗并且解决开发复杂度带来的挑战,提高开发和部署效率并且提升智能合约的代码质量,使得智能合约在部署到区块链网络上之前就能够保证编码和业务逻辑的正确性。
	

	% 同时应该注意到,空白页是故意留白,以便章节开头能够出现在偶数页。
	% 中文关键词。关键词之间用中文全角分号隔开,末尾无标点符号。
	\keywords{区块链;Hyperledger Fabric;智能合约;Node.js;IntelliJ插件}
\end{abstract}

%%%%%%%%%%%%%%%%%%%%%%%%%%%%%%%%%%%%%%%%%%%%%%%%%%%%%%%%%%%%%%%%%%%%%%%%%%%%%%%
% 论文的英文摘要
\begin{englishabstract}
	With the digital transformation continues to accelerate in recent years, all walks of life begin to pay attention to the way of data storage. 
	As a distributed data storage architecture, which is traceable, immutable, and decentralized, blockchain technology has set off a global boom. 
	Blockchain has entered a new stage of 3.0 nowadays. 
	Blockchain architecture represented by Hyperledger Fabric can not only serve the financial field, but also provide a new method to reduce the cost of trust and transaction for various industries such as regulatory technology, payment, and logistics. 
	At the same time, as the only intermediary between developers and blockchain networks, methods of developing smart contracts need more attention. 
	However, mainstream Integrated Development Environment (IDE) lacks support for developing smart contracts currently. 
	On the one hand, it is difficult to code smart contracts, which affects developing efficiency and code quality. 
    On the other hand, it is difficult to debug smart contracts in blockchain networks, since modification requires complex deployment process every time and there is a lack of tools to run and test smart contracts off-chain.
	
	Therefore, this thesis designs a scheme for Node.js smart contracts in Hyperledger Fabric, which facilitates its coding, debugging, and test. 
	It also implements a tool in the form of IDE plugin to help developers develop smart contracts off-chain. 
	This thesis develops the plugin based on IntelliJ platform and DevKit SDK, 
	which uses FreeMarker template technology to generate initial smart contracts and applies GPT-2 natural language processing model to provide code intelligent hints for Hyperledger Fabric smart contracts.
	The plugin implements simulation invocation and test of smart contracts by Hyperledger Fabric mock stub, Mocha framework and other technologies.
	It can be combined with IDE debugging mode and support breakpoints and single-step debugging if exception occurs during invocation of smart contracts.
	
	This thesis describes design and implementation of a plugin for smart contracts development, which is easy to use. 
	The plugin provides developers with a lightweight off-chain development tool for smart contracts. 
	It can not only reduce the cost of time when publishing smart contracts but also fix the challenge in the aspect of complexity of operation.
	The plugin improves the efficiency of development and deployment and enhances code quality of smart contracts. 
	It helps developers ensure correctness of code and business logic before publishing smart contracts in blockchain networks.
	
	% 英文关键词。关键词之间用英文半角逗号隔开,末尾无符号。
	\englishkeywords{Blockchain, Hyperledger Fabric, Smart Contracts, Node.js, IntelliJ Plugin}
\end{englishabstract}

%%%%%%%%%%%%%%%%%%%%%%%%%%%%%%%%%%%%%%%%%%%%%%%%%%%%%%%%%%%%%%%%%%%%%%%%%%%%%%%
% 论文的前言,应放在目录之前,中英文摘要之后,一般不需要
%
%\begin{preface}
%
%在过去的40年中,手写中文文本领域识别(HCTR)取得了很大的进展[1,2]。
%
%\vspace{1cm}
%\begin{flushright}
%饶安逸\\
%2018年5月15日于南大仙林
%\end{flushright}
%
%\end{preface}

%%%%%%%%%%%%%%%%%%%%%%%%%%%%%%%%%%%%%%%%%%%%%%%%%%%%%%%%%%%%%%%%%%%%%%%%%%%%%%%
% 生成论文目录
\tableofcontents

%%%%%%%%%%%%%%%%%%%%%%%%%%%%%%%%%%%%%%%%%%%%%%%%%%%%%%%%%%%%%%%%%%%%%%%%%%%%%%%
% 生成插图清单。如无需插图清单则可注释掉下述语句。
\listoffigures

%%%%%%%%%%%%%%%%%%%%%%%%%%%%%%%%%%%%%%%%%%%%%%%%%%%%%%%%%%%%%%%%%%%%%%%%%%%%%%%
% 生成附表清单。如无需附表清单则可注释掉下述语句。
\listoftables

%%%%%%%%%%%%%%%%%%%%%%%%%%%%%%%%%%%%%%%%%%%%%%%%%%%%%%%%%%%%%%%%%%%%%%%%%%%%%%%
% 开始正文部分
\mainmatter

%%%%%%%%%%%%%%%%%%%%%%%%%%%%%%%%%%%%%%%%%%%%%%%%%%%%%%%%%%%%%%%%%%%%%%%%%%%%%%%
% 学位论文的正文应以《绪论》作为第一章,本模板是按照自身功能模块组织的,并非论文中的章节安排

\chapter{绪论}

\section{项目背景}

互联网的诞生便于人们在金融、物联网、供应链、医疗等多种领域进行数据共享。
这些数据共享的场景通常是由多个组织或个人共同参与的,由于多方之间存在信任危机,因此互联网对于他们而言是存在安全隐患的。
自2008年中本聪首次提出区块链的概念以来~\cite{Bitcoin},区块链技术就为涉及利益交互的各种应用领域提供了一种解决信任问题的方式,
它可以让各个组织或个人制定自己的交易规则,并且所有的交易记录都是可以追踪溯源的。

区块链的各个节点持有完全一致的数据,交易各方之间不存在一个可以进行干涉的中心节点。
这种去中心化的架构配合底层的密码学技术~\cite{2016区块链技术发展现状与展望},使得数据无法篡改,保证了交易的安全性,如图~\ref{fig:1.1}所示。
区块链成为了存储数据的保险箱,这就是它最大的价值所在。

\begin{figure}[htb]
  \centering
  \includegraphics[width=0.85\linewidth]{figure/intro/区块链网络结构.png}
  \caption{区块链网络结构}\label{fig:1.1}
\end{figure}

对于区块链领域的开发工作而言,智能合约起到了关键性的作用,因为它是开发人员与区块链上的账本数据进行交互的唯一途径。
智能合约这一概念最早是由一名跨多领域的法律学者Nick Szabo于20世纪90年代提出的~\cite{王春宇2016智能合约与金融合约}。
随着以太坊为代表的面向公有链的开源平台不断出现,智能合约的执行有了可信环境~\cite{QianYang},开始应用在区块链领域。

智能合约和日常生活中的各种合约相似,都是基于各个参与者协商后的结果来制定一系列规则,以此约束某一场景下的行为。
智能合约作为业务逻辑的承载者,是运行在区块链分布式节点上的一段程序。
在区块链场景下,各个组织或者个人需要按照程序中规定的业务逻辑执行交易。
如今区块链技术不仅仅应用在金融领域,而是扩展到日常生活的方方面面,融入到传统行业中提供数字化转型的解决方案。

由Linux基金会主导发起的Hyperledger Fabric~\cite{Dhillon}作为面向联盟链的区块链开源平台,成为了企业级应用的首选。
Hyperledger Fabric这一架构利用独特的共识协议,使得交易的处理速度不再像以太坊的公有链那样受到较大的限制,由每分钟几百笔交易提升到了每分钟大约50万笔交易~\cite{DBLP:journals/access/PinnaIBTM19}。
智能合约的开发也得到了多种编程语言的支持,例如当前主流的开发语言Java、Go、Node.js等,为企业的开发人员提供了更多选择。

然而,区块链开发人员对于Hyperledger Fabric智能合约的编写主要还停留在使用文本编辑器的阶段,开发环境较为简陋,缺少IDE能够提供的代码提示功能,从而导致了智能合约开发效率低下~\cite{罗雄2020}。
这意味着开发人员的编码工作需要耗费更多时间,而且代码质量从基本的语法和上下文语义的角度都无法得到保证。
将这样漏洞频发的智能合约代码部署到区块链节点上会引发一系列灾难。

首先,在实际开发环境中,当智能合约在区块链网络节点上运行报错时,开发人员需要进行调试工作。
然而,当前主流的IDE没有提供针对智能合约调试或者测试相关的功能,使用简单的文本编辑器来修改出错的智能合约将耗费大量的时间和精力。

其次,区块链的优势之一就是它的不可篡改性,这为涉及利益交互的交易参与者提供了良好的安全保障,
但是同时,这一特点意味着对智能合约的每一次修改调试,都可能需要重新启动区块链网络,创建通道并且将节点加入通道,智能合约还需要经历安装、实例化等生命周期中的必要操作才能在区块链节点上运行。

再次,区块链实际应用场景和开发者部署一个单节点的测试网络不同,需要大量网络服务器节点,而每次对有漏洞的智能合约做出改动后都需要在每一个节点上进行上述操作~\cite{XiaozhouYang},这是一项冗余繁杂的工作。

即使开发人员编写了一份完美的智能合约,它能够在区块链网络节点上被SDK、CLI等客户端调用仍然需要具备一个前提——搭建并部署一个可以正常工作的区块链网络,而这些繁杂的步骤对于智能合约开发工作而言却不是一个必要条件。

因此,需要一个针对Hyperledger Fabric智能合约开发的辅助工具,能够方便开发人员在链下进行智能合约的开发,在将智能合约发布到区块链网络节点之前就确保其实现和业务逻辑的正确性,从而提升代码质量和部署效率。
本文基于以上项目背景研究解决方案,实现了一种面向Hyperledger Fabric的智能合约开发插件。

\section{国内外相关研究和应用现状}

区块链作为一个可溯源、不可篡改、去中心化的分布式账本,提供了一种有价值的机制可以让多方安全地共享数据并且降低成本。区块链技术从概念的提出到如今的广泛应用,其发展大致分为以下三个阶段。

“区块链1.0”阶段聚焦在可编程货币上,中本聪先生提出的比特币为互联网实现货币交易提供了可能性,越来越多的虚拟数字货币因为区块链技术而诞生。区块链开始在金融领域创建新的电子货币体系~\cite{DBLP:journals/hmd/Portmann18b},使得世界各地的人们都可以依靠互联网进行安全且便捷的买卖交易。

“区块链2.0”阶段引入了智能合约的概念~\cite{Sheikh},出现了以以太坊(Ethere- ~\linebreak um)为代表的区块链平台,区块链开始向股票、期权等金融衍生领域扩展~\cite{2015Blockchain}。以太坊通过智能合约让参与交易的组织或个人可以制定一份需要彼此严格遵守的协议,在这个交易过程中不再需要第三方中介。

“区块链3.0”阶段开始向日常生活的各行各业发展,只要是涉及信任问题的场景,都有区块链发挥价值的可能性。不同于面向公有链的以太坊,超级账本Hyperledger Fabric的出现为企业级的联盟链提供了平台,具有更高的商业价值。同时,交易效率也得到了极大提升。

如今区块链技术的价值愈加体现,国内外越来越多的专家学者开始涉入这一新兴领域进行学术研究~\cite{张荣2017, 陈旭2017},也有越来越多的企业将这一技术落地到实际项目中,而智能合约在区块链架构中扮演着不可或缺的角色,其开发也逐渐受到重视。

以太坊平台支持使用图灵完备语言进行智能合约编写,例如类似Java的高级编程语言Solidity、类似Python的脚本语言Serpent或者类似Assembly的低级语言LLL。
其中Solidity作为一种专门为以太坊智能合约设计的面向对象的高级编程语言,受到了C++、JavaScript等语言的启发,使用最为广泛,可以运行在以太坊虚拟机(Ethereum Virtual Machine,EVM)上。

区块链的实际应用场景可能涉及复杂的背景和用户需求,那么智能合约所承载的业务逻辑的体量也会随之增大,这样具有一定规模的智能合约程序代码已经不适合使用简单的文本编辑器进行开发,最好的方式就是使用一个能够提供编辑、编译、调试等功能的集成开发环境。

随着以太坊的发展,相关开源社区提供了一些智能合约的集成开发环境,例如Remix、Truffle等~\cite{QianYang},其中Remix作为以太坊官方推荐工具尤为主流。
Remix IDE是一款支持开发者在线使用浏览器进行智能合约开发的Web端工具,它针对Solidity语言编写的智能合约提供了代码编辑、编译、调试、测试等功能,这些功能模块都是以插件的形式存在的。
Remix IDE界面如图~\ref{fig:1.2}所示,由图标面板、侧方插件面板、主开发面板以及终端四部分组成。

\begin{figure}[htb]
  \centering
  \includegraphics[width=\linewidth]{figure/intro/Remix界面.png}
  \caption{Remix IDE开发界面}\label{fig:1.2}
\end{figure}

图标面板是Remix IDE支持的各个功能插件的入口,选中后图标对应的插件会出现在侧方插件面板中;侧方插件面板主要提供了插件独立的图形用户界面(Graphical User Interface,GUI),使得开发者完成相关配置内容或者填写具体参数;主开发面板是编写Solidity智能合约的区域,与传统IDE类似,在开发者编码的同时,给出代码补全、代码高亮、代码错误提示等内容;终端使得开发者能够和开发工具GUI进行交互,并且支持脚本的运行。

进入区块链3.0阶段后,企业操作系统(Enterprise Operating System,EOS)区块链平台成为了有力的竞争者之一。由于EOS智能合约的开发语言是C++,所以开发人员通常使用VScode IDE 作为EOS智能合约的开发工具。EOSFactory也提供了VScode IDE的EOS智能合约开发插件~\cite{罗雄2020},但其并没有针对运行、调试、测试方面的支持。

Hyperledger Fabric自2018年作为开源项目发布以来,成为了针对企业级联盟链的区块链架构代表。由于引入了身份认证的角色,各个组织或者个人可以规定数据公开范围,对联盟的创建提供了友好的机制,并且保证了数据的私密性和安全性。
相比以太坊,Hyperledger Fabric在智能合约上也做了很多改进和扩展工作,使得开发人员可以使用Java、Go、Node.js等多种当前主流的开发语言进行智能合约编码。
Hyperledger Fabric具有可配置、模块化、性能高等众多优势,但是该技术较新且仍然在发展阶段。
因此,选择Hyperledger Fabric作为本文研究对象是具有一定必要性的。

目前,Hyperledger Fabric已经从最初的1.x版本升级到了2.x版本,新版本使其具有更多的优势。加强了对于智能合约的去中心化管理;针对联盟链内的合作和共识提供了新的链码应用模式;添加了新功能使得数据私密性得到了更强的保障等。从智能合约的编码角度考虑,ACE作为一款高性能的Web编辑器,可以支持120多种语言的语法高亮~\cite{罗雄2020},ACE代码编辑器基于脚本语言JavaScript开发,支持与JavaScript应用或者Web页面结合使用。然而,在Hyperledger Fabric智能合约的开发上却缺少一个针对性的集成开发环境,用以提供编辑、运行、调试、测试等重要功能,因此区块链开发者仍然面临开发效率、代码质量带来的挑战。

\section{解决方案及创新}

由于区块链网络的搭建和部署工作较为复杂,而且针对智能合约的开发而言,区块链网络及其相关内容并不是必要条件,因此本文研究并提出了一种无需区块链网络的链下智能合约开发工具。

同时,考虑到可扩展性和易用性,该工具作为一个基于IntelliJ平台的IDE插件,提供轻量级的开发辅助支持。为了让开发者在区块链网络上发布智能合约之前就可以保证其正确性,该插件需要实现以下几点功能:

\begin{itemize}
    \item 智能合约初始模板生成:对于智能合约的编写,Hyperledger Fabric 1.x版本提供了基于fabric-shim包的开发方式,而升级后的2.x版本增添了新的基于fabric-contract-api的开发方式,插件能够让用户根据不同场景下的需求,选择创建合适的初始智能合约模板,提高开发效率。
    \item 智能合约代码智能提示:在程序代码编写过程中,给出实时的智能代码补全提示,降低开发门槛并且提升编码速度。
    \item 智能合约模拟调用与测试:开发人员完成智能合约编写工作后,可以进行接口的模拟调用,获取调用结果。
\end{itemize}

\section{主要工作}

本文以软件工程的思想作为根本,经过需求分析、系统整体设计和模块详细设计,并基于以上给出了解决方案的具体实现和测试。
本文指出了当前区块链领域中智能合约开发存在的问题并完成了面向Hyperledger Fabric的智能合约开发插件的设计与实现:

\begin{itemize}
    \item 首先,本文详细地介绍了开发这款插件的项目背景。通过对区块链领域研究现状的描述与分析,引入了目前区块链开发人员在Hyperledger Fabric智能合约开发过程中的问题和难点。阐述了提供一种链下辅助工具的意义、重要程度及其优势。
    \item 本文简洁地介绍了开发这款插件所涉及的关键技术,包括区块链领域的相关技术、集成开发环境中插件的相关技术以及自然语言处理的相关技术。
    \item 本文深入地分析了面向Hyperledger Fabric智能合约的开发插件的功能性需求和非功能性需求,将插件切分成不同的模块进行需求的描述和分析,并基于需求分析进行项目的整体架构设计,由此对插件各个子模块进行详细的设计。
    \item 本文根据项目的需求分析和系统设计进行了具体的代码实现,给出了对于创建初始智能合约、智能合约代码提示、智能合约模拟调用等重要功能的关键代码。
    \item 最后,本文展示了对于插件各个功能的测试情况,包括测试所用的技术和具体测试流程,并给出了测试结果。本文结合测试结果总结了所述解决方案的有效性和不足,提出了后续优化思路。
\end{itemize}

\section{组织结构}

本文的组织结构如下:

第一章,绪论部分。
介绍了项目的背景和相关应用发展现状,阐明面向Hyperledger Fabric的智能合约开发插件所能解决的问题和解决方案,并简要列举了本文的工作内容。

第二章,技术综述。
对这一智能合约辅助工具的开发所涉及的关键技术给出了简洁的介绍,包括区块链、Hyperledger Fabric、IntelliJ插件、自然语言处理及其相关技术等,并且解释了选择该技术解决问题的理由。

第三章,插件的需求分析与设计。
对插件的功能需求和非功能性需求进行了细致分析。
简洁阐明了插件总体架构以及各个模块的划分思路,并且详细描述了插件各个模块的设计方案。

第四章,插件的实现。
基于第三章的需求分析和设计,描述了智能合约开发插件的实现细节,展示了插件核心功能的关键代码及其插件运行的界面图。

第五章,插件的测试。
对插件的功能进行测试并对整体运行进行性能测试,展示并分析测试结果。

第六章,总结与展望。对本论文的工作和成果做出总结,并且提出了智能合约开发插件的可优化思路,对插件的可扩展点做了进一步的展望。


\chapter{技术综述}

本章对于面向Hyperledger Fabric智能合约的开发插件所应用到的关键技术给出简要介绍。插件开发主要采用Java编程语言,智能合约代码提示的部分应用了Python脚本语言,由于此插件针对Node.js智能合约,因此在开发过程中还使用了JavaScript脚本语言。项目开发过程涉及本章将要介绍的区块链、Hyperledger Fabric、IntelliJ插件、自然语言处理、网页爬虫、Flask Web应用框架、Freemarker模板等相关技术。

\section{区块链技术}

\subsection{核心架构}

区块链采用链式区块存储交易数据,区块包含区块头和区块体两部分,区块数据结构如图~\ref{fig:2.1}所示。链上的每一个区块头都会保存前一个区块头的哈希值,以此形成可追踪溯源且不可篡改的链条结构~\cite{Bitcoin},区块利用了Merkle树结构记录交易信息~\cite{DBLP:reference/db/Carminati18b}。

区块链是一种去中心化的分布式网络,其底层是一个与传统模式不同的分布式数据库,它通过共识机制、P2P网络、加密算法等技术实现了一个不可篡改、可溯源、安全可靠的网络~\cite{XiliangWang},如图~\ref{fig:2.2}所示。

\begin{figure}[htb]
  \centering
  \includegraphics[width=\textwidth]{figure/technology/区块数据结构.png}
  \caption{区块数据结构}\label{fig:2.1}
\end{figure}

\begin{figure}[htb]
  \centering
  \includegraphics[width=0.7\textwidth]{figure/technology/区块链架构.png}
  \caption{区块链架构}\label{fig:2.2}
\end{figure}

\subsection{共识机制}

共识机制是区块链技术保证数据正确性和一致性的关键,它作为一种设定好的协议,使得各个节点之间进行规范化的数据验证,从而确保整个分布式网络达成共识。

不同的区块链平台采用的共识机制不尽相同,比特币架构采用了工作量证明(Proof-of-Work,PoW)的共识机制~\cite{XuanHan},该机制依赖网络中各个节点的计算能力来确保整个分布式网络的数据一致性,算力强的节点获得投票权,并且安全性得到了加强,攻击者需要掌控一半以上的算力才可以攻击网络;
以太坊通常采用PoW、权益证明机制(Proof-of-Stake,PoS)~\cite{DBLP:conf/iceccs/ThinDBD18}以及股份授权证明机制(Delegated-Proof-of-Stake,DPoS)~\cite{YongZhang}的混合性共识机制,PoS改进了PoW共识机制,通过减少无用计算提升了资源利用率并且提高了区块链网络性能,但网络的去中心化则被削弱,而DPoS弥补了PoW和PoS的缺陷,引入证明人机制在降低资源消耗的同时保证了去中心化的优势;
Hyperledger Fabric 1.0采用了特有的Solo和Kafka机制,2.0版本则引入了Raft等共识机制~\cite{ShaoqiWang, ZhikeWang},确保交易中的所有参与者都必须经过认证。

\subsection{智能合约}

智能合约最开始并不属于区块链领域,Nick Szabo最早定义了智能合约的概念:一个智能合约是一套以数字形式定义的承诺,包括合约参与方可于其上执行这些承诺的协议~\cite{DBLP:journals/fgcs/ZhengXDCCWI20}。区块链的出现为智能合约的执行提供了可信环境,随着以太坊平台的出现,智能合约初次被应用在区块链领域。

智能合约本质上是一段运行在区块链网络节点上的程序代码,其在区块链网络中承载了预先规定好的业务逻辑,包括对象的状态、交易触发的条件、数据更新的规则等,开发人员只有通过智能合约才能获取或者更新区块中存储的数据。
以太坊平台的智能合约是不可变的,其一旦部署在网络节点上,就无法再进行修改~\cite{DBLP:journals/corr/abs-1807-01868}。

当交易的参与节点满足预先规定的条件时,智能合约将被节点执行,相应的业务逻辑代码将被调用。
在区块链网络中,作为交易参与者的每个节点都必须严格执行智能合约,其执行环境可以是像Hyperledger Fabric采用的Docker容器~\cite{DBLP:journals/sigops/Boettiger15},也可以是像以太坊平台采用的EVM虚拟机~\cite{DBLP:journals/corr/abs-1710-06372},这两种执行环境都保证了智能合约执行时所需要的资源的独立性和隔离性。

智能合约的开发语言随着区块链的发展也产生了变化。区块链2.0阶段的代表平台以太坊,其智能合约开发语言主要采用Solidity,Solidity是一种专门为以太坊智能合约设计的编程语言。进入区块链3.0阶段后,智能合约开发语言逐渐向简洁的高级程序开发语言转变,例如EOS平台采用的C++以及Hyperledger Fabric平台采用的Node.js、Java、Python、Go等多种编程语言。多种编程语言对智能合约的支持降低了区块链开发的门槛。

\section{Hyperledger Fabric平台}

\subsection{核心概念}

Hyperledger Fabric是区块链进入3.0发展阶段的典型代表平台之一,其交易效率得到大幅度提升,支持组件模块化且可配置~\cite{DBLP:conf/eurosys/AndroulakiBBCCC18},为企业级应用的联盟链提供了有价值的选择~\cite{DBLP:conf/isads/XuZZP17}。
Hyperledger Fabric平台引入了以下特有的一些概念~\cite{NingDong}:

\begin{itemize}
    \item 通道(Channel):为联盟链提供了数据隔离方面的支持,是Hyperledger Fabric的一个独立的子网。节点需要在满足安全策略的前提下才能加入通道,而只有加入了通道的节点才有资格接触数据,因此解决了隐私问题。每个节点也可以加入不同通道以达到不同的交易目的。
    \item 世界状态(World State):交易数据当前的状态,以键值对形式存在。
    \item 账本(Ledger):区块链网络存储的数据,包含世界状态以及区块链交易日志。账本具有只可添加、不可修改的特性,保证了数据的有效性和一致性,使得交易安全且不可篡改。区块链网络节点上可以存储多个账本。
    \item 背书(Endorse):制定了一种策略,区块链网络节点需要按照背书策略进行数据验证并且达成共识,从而完成有效的交易。
    \item 成员服务提供者(Membership Service Provider,MSP):用于区块链网络节点的身份认证,负责管理节点的身份信息,关联了网络实体与加密标识,采用传统的公钥基础设施(Public Key Infrastructure,PKI)分层模型。
\end{itemize}

\subsection{架构设计}

Hyperledger Fabric网络架构由Peer节点、Orderer节点、证书授权(Certificate Authority,CA)节点组成:

\begin{itemize}
    \item Peer节点:存储了账本和区块,负责智能合约的执行。其中Committing Peer节点会接受生成的区块,经过验证的区块将提交给账本;Endorsing Peer节点安装了智能合约,将参与背书过程,进行交易的模拟执行;一部分Peer节点;Leader Peer负责与Orderer节点进行沟通;Anchor Peer可以有零至多个,用于处理跨组织的节点通信。
    \item Orderer节点:组织中可以有多个Orderer节点,其主要负责接收并验证签名和交易~\cite{DBLP:conf/dsn/SousaBV18},通过排序操作在交易顺序上建立共识,创建区块并向Peer节点广播区块。Hyperledger Fabric的排序机制可以选择单一节点的Solo、多节点的Kafka或者Raft等协议。
    \item CA节点:CA节点作为区块链网络的认证中心,扮演了生成并颁发证书、创建账户的角色。CA节点使得Hyperledger Fabric网络中的每个操作都具有数字签名,智能合约可以根据数字签名判断是否执行、如何执行,该节点保证了Hyperledger Fabric网络的安全性。
\end{itemize}

以上几种节点都有各自的角色,负责不同的任务,Hyperledger Fabric的整体节点架构如图~\ref{fig:2.3}所示。

\begin{figure}[htb]
  \centering
  \includegraphics[width=\linewidth]{figure/technology/节点架构.png}
  \caption{节点架构}\label{fig:2.3}
\end{figure}

\subsection{Hyperledger Fabric智能合约}

智能合约在Hyperledger Fabric架构中被称为链码(Chaincode)。
链码作为一段程序代码运行在Peer节点上,提供了业务逻辑的接口。
只有安装了链码的Peer节点才可以参与背书过程。
客户端应用程序可以通过调用链码实现对区块链账本的查询、更新操作。

目前Hyperledger Fabric已经支持Node.js、Go、Java等多种编程语言进行链码开发。
区块链开发人员在完成智能合约的编写工作之后需要进行测试工作,针对Hyperledger Fabric链码的测试方式有两种,一种是搭建实际的区块链网络,在环境中切入开发者模式,通过调用智能合约接口进行测试;
另一种是利用Mockstub编写单元测试代码。

智能合约在被客户端应用程序调用之前,需要经历安装、实例化等操作。
链码的生命周期要求通道中的各个组织就链码达成一致,这一过程包括以下几个步骤:

\begin{itemize}
    \item 组织需要将链码打包。
    \item 需要进行背书交易或者对账本进行访问的组织必须在Peer节点上安装链码。
    \item 区块链网络中的组织需要根据背书策略就链码定义达成共识,背书策略将指定批准交易的组织数量。
    \item 将链码定义提交到通道上进行链码部署。
\end{itemize}

链码将会运行在Docker 容器中,以此避免链码漏洞对整体网络造成大范围影响。
链码的概念模型如图~\ref{fig:2.4}所示。

\begin{figure}[htb]
  \centering
  \includegraphics[width=0.9\linewidth]{figure/technology/链码概念模型.png}
  \caption{链码概念模型}\label{fig:2.4}
\end{figure}

\section{IntelliJ插件}

\subsection{Action系统}

Action系统是IntelliJ的集成开发环境所提供的扩展点之一~\cite{DBLP:conf/sigsoft/HajjN20},开发IntelliJ插件时可以利用Action作为插件功能对外暴露的方式。
Action既可以被添加到IDE的菜单栏或者工具栏,也可以通过快捷键或搜索的途径调用。
插件项目自定义的Action需要继承Action系统提供的AnAction作为父类,通过重写~\texttt{actionPerformed()}方法实现具体的业务逻辑,可以获取该方法的AnActionEvent参数,以此操作IDE中的项目、编辑窗口、文件等元素。实现业务逻辑的Action还需要进行注册才能决定如何显示在IDE中。

\subsection{用户接口组件}

IntelliJ平台提供的插件SDK包含了多种用户接口组件(User Interface Component),User Interface Component作为IntelliJ自定义的Swing组件,为插件开发者提高了开发效率。
在实现插件提供的业务逻辑功能后,可以通过这些特殊的组件将功能与IDE UI相结合。

本论文采用了Tool Windows、Messages等组件以辅助插件的开发。
Tool Windows相当于IDE的独立子窗口,当前主流项目构建工具Ant、Gradle、Maven等都有其自定义的Tool Window以提供功能入口。
Messages可以帮助显示消息盒子,插件使用者可以以此方式与插件进行交互。

\subsection{DevKit插件}

DevKit插件是IntelliJ平台所提供的一种开发插件的便捷方式,开发者可以利用DevKit自定义的SDK和Actions进行插件的开发。
在使用DevKit之前,需要在IDE中安装该插件,并且配置好IntelliJ平台的SDK。进行业务逻辑开发后可以像传统项目一样运行并调试插件。完成开发测试的插件可以进行构建和安装,开发者可以选择将其部署在本地IDE或者发布到插件仓库中供他人安装和使用。

\section{自然语言处理}

\subsection{自然语言处理概述}

近年来,自然语言处理(Natural Language Processing,NLP)在人工智能领域的应用范围逐渐扩大。
NLP最早广泛用于机器翻译,乔治敦大学联合IBM于1954年研发了第一台NLP机器~\cite{DBLP:conf/icdm/AbeT10},该机器能够将俄语句子翻译成英语句子。
现如今,NLP涉足了越来越多的应用领域,语音识别、舆情监测、文本生成等方向都可以采用NLP技术。
由此可见,NLP主要针对词语、句子、段落、文章进行处理。
随着深度学习的盛行,经历了基于规则和统计方法的开发者逐渐将深度学习应用于NLP中~\cite{DBLP:conf/nips/SutskeverVL14},而这意味着良好的数据集将对处理结果起到至关重要的作用。

\subsection{Transformer模型}

Google团队在2017年首次提出了Transformer模型~\cite{DBLP:conf/nips/VaswaniSPUJGKP17},该模型摒弃了之前广泛使用的卷积神经网络和循环神经网络,引入了自注意力机制和多头注意力机制来处理文本,以此提升处理速度、效果和计算资源的利用率。Transformer模型包括负责接收文本作为输入的编码器以及负责把新表达转换为目的词的解码器。

\subsection{GPT-2模型}

GPT-2模型发布于2019年,是目前文本生成模型的最佳选择之一,用于训练该模型的数据集体量相比在它之前的Transformer、BERT~\cite{DBLP:conf/naacl/DevlinCLT19}等模型都要庞大得多,采用GPT-2模型生成的文本具有更好的连贯性和准确度。本论文利用该模型实现针对Hyperledger Fabric智能合约的代码智能补全。

GPT-2内部采用了Transformer模型的解码器并且仍然沿用了自注意力机制,在处理单词序列并传入神经网络之前融入了对上下文单词的理解。每一层解码器都会处理输入进来的单词序列,并将处理后的结果传入下一层解码器,模型会利用最终向量求得词汇表中单词的注意力得分,得分高的作为输出结果。

\section{网页爬虫}

网页爬虫是一段可以自动抓取网页资源的脚本程序,允许根据不同目标对网页进行分析,过滤并锁定数据,从而进行数据爬取。网络爬虫基于网页的URL,在抓取过程中不断获取新的URL放入待抓取队列,直到满足程序定义的终止条件。

由于数据集的体量对于NLP模型训练和最终的文本生成效果具有决定性的影响,因此本论文需要采集大量的Hyperledger Fabric智能合约,数据采集通过Python网络爬虫实现。

\section{Flask框架}

Flask框架是Python开发者常用的一款轻量级的Web框架~\cite{DBLP:books/daglib/0034106}。本插件采用IntelliJ IDEA作为IDE,其底层开发语言为Java,因此插件主要采用Java进行编码,而智能合约的代码提示功能采用了NLP技术,其涉及的数据采集、数据清洗、模型训练等流程均采用Python实现,因此本插件利用Flask Web框架对外提供NLP服务的接口,既提升了代码补全速度,也使得两种语言得到更好的解耦。

\section{FreeMarker模板}

FreeMarker作为一款模板引擎,为开发者提供了一种基于模板预定规则生成输出文本的工具~\cite{DBLP:journals/program/RadjenovicMS09},一般用于HTML网页、电子邮件、源代码、配置文件的生成。

模板内容的编写采用特殊且简单的FreeMarker模板语言(FreeMarker Template Language,FTL)。
该技术让开发者在模板中专注文本的展示形式,在模板外专注文本的展示内容。
本文利用FreeMarker模板减少了大量冗余代码,提升了开发效率。

FreeMarker引擎有多种模板加载方式,开发者可以通过内建模板加载器加载指定文件路径、类路径或Web应用目录下的模板,还可以根据不同需求从多个位置或其他资源加载模板。

\section{本章小结}

本章对面向Hyperledger Fabric智能合约的开发插件在研发过程中所采用的关键技术进行了简要介绍,包括其概念、优势、基础原理等。插件基于IntelliJ平台的集成开发环境和DevKit SDK进行搭建,利用FreeMarker模板技术实现Fabric初始智能合约的生成;采用Mockstub配合Node.js测试框架完成对智能合约的模拟调用和测试功能;通过NLP技术提供智能合约的代码提示。同时,本章阐述了选择上述技术实现目标功能的理由。


%\chapter{插件的需求分析}

\section{功能性需求分析}

\subsection{功能需求概述}

面向Hyperledger Fabric智能合约的开发插件作为辅助区块链开发人员进行智能合约开发的工具,需要满足智能合约开发周期涉及的创建、编写、调用以及测试等功能性需求,插件的用例图如图~\ref{fig:3.1}所示:

\begin{itemize}
  \item 初始代码生成,智能合约开发者需要根据Hyperledger Fabric的使用版本和应用场景选取合适的智能合约初始模板,代码生成模块将为开发者创建一份初始智能合约。
  \item 代码智能提示,开发者在IDE中编写智能合约代码时,需要插件提供实时的代码智能提示,该功能需要补全当前键入的单词序列,并给出备选提示。
  \item 模拟调用及测试,当开发者完成智能合约编写后,需要插件提供工具窗口,展示当前项目中的智能合约及其内部的接口信息,开发者可以通过该窗口完成智能合约的模拟调用并获取测试结果。
\end{itemize}

\begin{figure}[htb]
  \centering
  \includegraphics[width=5in]{figure/requirement/用例图.png}
  \caption{用例图}\label{fig:3.1}
\end{figure}

\subsection{Fabric智能合约初始模板生成模块}

Hyperledger Fabric 1.0正式版本于2017年发布,在IBM、Google等企业的大力支持下,其2.0正式版本于2020年发布.
升级后的Fabric主要增加了以下几点:

\begin{itemize}
  \item 智能合约去中心化管理。
  \item 用于协作和共识的新链码模式。
  \item 外部的链码启动程序。
  \item 新的共识机制。
  \item 新的智能合约开发包。
\end{itemize}

本论文经调研发现,虽然新版本的Hyperledger Fabric具有很多新优势,但是相比1.0版本其稳定性有所下降,而且由于目前新版本发布时间较短,其相关研究和应用程序较少,所以仍然有不少企业在使用1.0版本。

Fabric 1.0版本对智能合约的开发提供了fabric-shim包,fabric-shim是较为底层的智能合约开发包,它封装了Hyperledger Fabric的网络通信协议gRPC,使得开发者可以直接调用链码接口。基于fabric-shim开发的智能合约需要定义一个类,在这个类中需要实现两个预定义的方法,分别是~\texttt{Init()}方法和~\texttt{Invoke()}方法,这两个方法都会接收节点传入的stub作为参数,通过stub可以进行账本的查询、更新等操作。在部署完毕的区块链网络中进行链码初始化时,客户端应用程序会调用~\texttt{Init()}方法,而客户端应用程序对智能合约接口的调用将会转化为对~\texttt{Invoke()}方法的调用。
基于fabric-shim包进行智能合约编写较为简洁易用,适合业务逻辑简单的应用场景,但实现业务逻辑较为复杂应用场景则需要开发者在~\texttt{Invoke()}方法的实现中进行方法路由。

Fabric 2.0版本增加了一种新的链码开发包fabric-contract-api,该开发包提供了对智能合约更高层级的封装,使得开发者可以直接继承Contract类进行智能合约开发,减少了方法路由所产生的繁杂冗余代码。

由于Hyperledger Fabric两个版本各有优势劣势,其提供的智能合约开发包的适用场景也不尽相同,因此插件针对两种智能合约开发方式实现了初始模板生成的功能。插件需要IDE能够弹出提示窗口帮助用户创建初始智能合约,并在用户未创建项目环境等异常情况下给出错误警告和提示信息。

\subsection{Fabric智能合约代码智能提示模块}

随着区块链进入3.0可编程社会阶段,金融、物联网、供应链、医疗等多种领域都开始引入Hyperledger Fabric作为企业联盟链的区块链平台,为其提供数据存储和交易等方面的安全保障。智能合约所承载的业务逻辑逐渐复杂起来,代码体量将愈发庞大。因此,智能合约的编写工作不再适合使用简单的文本编辑器,否则对于区块链开发者而言会增大难度,也会浪费时间精力。

IntelliJ IDE作为当前主流的程序集成开发环境,对项目构建、代码编写、程序编译、运行、调试等需求提供了强大的支持,极大地提高了程序的开发效率。代码编辑器是IntelliJ IDE的重要组件之一,从代码编写角度实现了全面的功能,包括代码自动补全、语法高亮、错误提示等。

因为Hyperledger Fabric智能合约可以使用Java、Go等高级程序语言进行开发,所以开发者适合选择IntelliJ IDEA、GoLand等IDE代替普通的文本编辑器进行代码开发。

运行在Peer节点上的智能合约是客户端应用程序与区块链网络中的账本进行交互的唯一途径。与普通的应用程序相比,智能合约开发者在编码过程中不仅需要学习编程语言的特性和语法规则,还需要考虑智能合约代码的特殊设计以及规范性方面的要求,而这需要开发者熟悉Hyperledger Fabric平台对于智能合约所提供的对外接口。因此,插件需要在IntelliJ IDE代码补全功能的基础之上做进一步的增强效果,针对Hyperledger Fabric两个版本的智能合约编写提供代码智能提示功能。

IntelliJ IDE本身提供的代码补全效果是以键入字符序列时自动弹出的下拉列表的形式展现的。插件需要在开发者键入字符序列的同时,实时地展示代码智能提示,将GPT-2模型预测的Top-k选项加入下拉清单中作为备选提示,开发者可以根据编码目的选择最合适的选项。

插件需要基于IntelliJ平台提供的DevKit SDK开发包进行开发,在不影响IDE性能的前提下,将代码智能提示功能与IDE原有的代码补全效果相融合。

插件针对Hyperledger Fabric智能合约提供的代码智能提示功能可以降低智能合约的开发门槛,提升智能合约的开发效率,与此同时,让开发者将注意力集中在业务逻辑的实现上。

\subsection{Fabric智能合约模拟调用模块}

Hyperledger Fabric在网络搭建层面既支持单机上的单个或多个节点,也支持多机多节点部署。客户端应用程序在调用智能合约接口以对区块链账本进行增删改查等逻辑行为之前,至少需要在本地成功搭建一个单节点的Fabric网络。然而,Hyperledger Fabric的搭建过程较为复杂,而且在Fabric升级到2.0版本后,其平台搭建过程与1.0版本相比产生了较大的区别,智能合约的生命周期也发生了改动。

由于区块链网络的搭建繁杂耗时,且仅针对智能合约的开发而言不是必要条件,因此插件需要提供一个在无需搭建区块链网络的前提下就能对智能合约进行接口模拟调用的入口。

在实际开发环境中,由于开发者缺少一个针对Hyperledger Fabric智能合约的开发工具,因此在网络中部署的智能合约漏洞频发,导致其无法在Peer节点被执行。当智能合约执行报错时,开发者需要搭建并配置Dev模式进行智能合约调试,每一次修改智能合约后,仍然需要重新启动整个Fabric网络,创建通道,将参与交易的Peer节点加入通道,编译、打包、安装并实例化智能合约,开发者需要经历一系列复杂且冗余的操作后,才能重新尝试执行智能合约。
Hyperledger Fabric智能合约部署流程如图~\ref{fig:3.2}所示。

\begin{figure}[htb]
  \centering
  \includegraphics[width=5in]{figure/requirement/智能合约部署流程.png}
  \caption{智能合约部署流程}\label{fig:3.2}
\end{figure}

安全性是区块链技术最关注的因素之一,而网络整体的安全程度很大程度上取决于网络内的节点数量,在多节点的网络内进行智能合约的反复调试和反复部署更加费时费力。

基于IntelliJ平台开发的IDE没有针对Hyperledger Fabric智能合约调试或者测试提供相关的功能,因此,本插件作为一个Hyperledger Fabric智能合约的链下开发工具,需要为开发者提供可以进行智能合约模拟调用的工具窗口,并对测试结果进行展示。

智能合约开发者可以利用本插件的模拟调用及测试功能,在将智能合约发布到区块链网络节点之前就确保其实现和业务逻辑的正确性,既提升了代码质量,又能帮助开发人员节省大量时间,提升开发效率。

\section{非功能性需求分析}

本论文设计并实现的插件需要嵌入到IntelliJ平台的IDE中,所以对其非功能性需求较高。
插件性能需要满足一定标准,在不影响IDE其他功能正常使用的同时,改善开发者使用插件的体验感。
插件的性能、稳定性等重要的非功能性衡量指标如表~\ref{table:requirement}所示:

\begin{table}[htb]\scriptsize
\centering
\caption{非功能性需求列表}
\vspace{2mm}
% l - left, r - right, c - center. | means one vertical line
\begin{tabular}{ccc}
\toprule
\textbf{非功能性需求}&\textbf{目标}&\textbf{优先级}\\
\midrule
\textbf{可扩展性}&支持IntelliJ平台扩展点&高\\ \hline
\textbf{可用性}&良好的界面交互设计&高\\ \hline
\textbf{稳定性}&99.99\%的场景稳定运作&高\\ \hline
\multirow{2}*{\textbf{性能}}&对接口的调用的平均响应时间小于50ms;&\multirow{2}*{高}\\
~&智能合约的代码实时提示展示平均在30ms以内&~\\ \hline
\textbf{可适配性}&支持嵌入多种基于IntelliJ平台开发的IDE&中\\ \hline
\textbf{可部署性}&支持本地构建启动和仓库发布&中\\ \hline
\textbf{可重用性}&具有重用插件扩展点和业务逻辑的能力&中\\ \hline
\textbf{耦合性}&插件与自然语言处理服务解耦&中\\
\bottomrule
\end{tabular}
\label{table:requirement}
\end{table}


插件作为IntelliJ IDE的可插拔式开发辅助工具,需要具备较好的可扩展性。插件的项目结构清晰,且基于IntelliJ平台的DevKit进行开发,可以通过类的继承、配置文件等方式进行功能点或业务逻辑的扩充。

插件需要通过对话框、工具窗口等形式达到良好的可用性,便于开发者的理解和操作,易于上手。

插件需要设计良好的异常处理机制,且稳定地提供NLP服务以维护智能合约代码智能补全的实时效果,具有较高的可靠性。

插件需要进行良好的本地构建和仓库发布,以确保与IntelliJ平台的多种集成开发环境能够兼容适配,便于部署。

IntelliJ平台的插件开发除了可以基于DevKit SDK,还可以基于GitHub Template、Gradle或Maven等构建工具,本插件的功能模块可以快速便捷地迁移到其他方式开发的插件中。

\section{本章小结}

本章从功能性需求和非功能性需求的角度对面向Hyperledger Fabric智能合约的开发插件进行细致的分析。通过用例图展示了插件的功能性需求,并分别对智能合约初始模板生成、代码智能提示、模拟调用及测试等功能进行需求分析;非功能性需求从插件的性能、兼容、有效性等方面考虑并进行阐述。为下一章插件的概要设计做好了铺垫。

\chapter{插件的需求分析与设计}

本章主要对面向Hyperledger Fabric的智能合约开发插件进行需求分析和设计。
首先按模块进行功能性需求分析并介绍了插件的非功能性需求;
接着概述了插件的整体架构以及模块划分设计;
最后针对每个功能模块详细阐述设计思路。

\section{功能性需求分析}

\subsection{功能需求概述}

插件的用例图如图~\ref{fig:3.1}所示。

\begin{figure}[htb]
  \centering
  \includegraphics[width=0.9\linewidth]{figure/requirement/用例图.png}
  \caption{用例图}\label{fig:3.1}
\end{figure}

面向Hyperledger Fabric的智能合约开发插件作为辅助区块链开发人员进行智能合约开发的工具,需要满足智能合约开发周期涉及的创建、编写、调用以及测试等功能性需求:

\begin{itemize}
  \item 初始代码生成,智能合约开发者需要根据Hyperledger Fabric的使用版本和应用场景选取合适的智能合约初始模板,代码生成模块将为开发者创建一份初始智能合约。
  \item 代码智能提示,开发者在IDE中编写智能合约代码时,需要插件提供实时的代码智能提示,该功能需要补全当前键入的单词序列,并给出备选提示。
  \item 模拟调用及测试,当开发者完成智能合约编写后,需要插件提供工具窗口,展示当前项目中的智能合约及其内部的接口信息,开发者可以通过该窗口完成智能合约的模拟调用并获取测试结果。
\end{itemize}

\subsection{Fabric智能合约初始模板生成模块}

Hyperledger Fabric 1.0正式版本于2017年发布,在IBM、Google等企业的大力支持下,其2.0正式版本于2020年发布.
升级后的Fabric主要增加了以下几点:

\begin{itemize}
  \item 智能合约去中心化管理。
  \item 用于协作和共识的新链码模式。
  \item 外部的链码启动程序。
  \item 新的共识机制。
  \item 新的智能合约开发包。
\end{itemize}

本论文经调研发现,虽然新版本的Hyperledger Fabric具有很多新优势,但是相比1.0版本其稳定性有所下降,而且由于目前新版本发布时间较短,其相关研究和应用程序较少,所以仍然有不少企业在使用1.0版本。

Fabric 1.0版本对智能合约的开发提供了fabric-shim包,fabric-shim是较为底层的智能合约开发包,它封装了Hyperledger Fabric的网络通信协议gRPC,使得开发者可以直接调用链码接口。基于fabric-shim开发的智能合约需要定义一个类,在这个类中需要实现两个预定义的方法,分别是~\texttt{Init()}方法和~\texttt{Invoke()}方法,这两个方法都会接收节点传入的stub作为参数,通过stub可以进行账本的查询、更新等操作。在部署完毕的区块链网络中进行链码初始化时,客户端应用程序会调用~\texttt{Init()}方法,而客户端应用程序对智能合约接口的调用将会转化为对~\texttt{Invoke()}方法的调用。
基于fabric-shim包进行智能合约编写较为简洁易用,适合业务逻辑简单的应用场景,但实现业务逻辑较为复杂应用场景则需要开发者在~\texttt{Invoke()}方法的实现中进行方法路由。

Fabric 2.0版本增加了一种新的链码开发包fabric-contract-api,该开发包提供了对智能合约更高层级的封装,使得开发者可以直接继承Contract类进行智能合约开发,减少了方法路由所产生的繁杂冗余代码。

由于Hyperledger Fabric两个版本各有优势劣势,其提供的智能合约开发包的适用场景也不尽相同,因此插件针对两种智能合约开发方式实现了初始模板生成的功能。插件需要IDE能够弹出提示窗口帮助用户创建初始智能合约,并在用户未创建项目环境等异常情况下给出错误警告和提示信息。

\subsection{Fabric智能合约代码智能提示模块}

随着区块链进入3.0可编程社会阶段,金融、物联网、供应链、医疗等多种领域都开始引入Hyperledger Fabric作为企业联盟链的区块链平台,为其提供数据存储和交易等方面的安全保障。智能合约所承载的业务逻辑逐渐复杂起来,代码体量将愈发庞大。因此,智能合约的编写工作不再适合使用简单的文本编辑器,否则对于区块链开发者而言会增大难度,也会浪费时间精力。

IntelliJ IDE作为当前主流的程序集成开发环境,对项目构建、代码编写、程序编译、运行、调试等需求提供了强大的支持,极大地提高了程序的开发效率。代码编辑器是IntelliJ IDE的重要组件之一,从代码编写角度实现了全面的功能,包括代码自动补全、语法高亮、错误提示等。

因为Hyperledger Fabric智能合约可以使用Java、Go等高级程序语言进行开发,所以开发者适合选择IntelliJ IDEA、GoLand等IDE代替普通的文本编辑器进行代码开发。

运行在Peer节点上的智能合约是客户端应用程序与区块链网络中的账本进行交互的唯一途径。与普通的应用程序相比,智能合约开发者在编码过程中不仅需要学习编程语言的特性和语法规则,还需要考虑智能合约代码的特殊设计以及规范性方面的要求,而这需要开发者熟悉Hyperledger Fabric平台对于智能合约所提供的对外接口。因此,插件需要在IntelliJ IDE代码补全功能的基础之上做进一步的增强效果,针对Hyperledger Fabric两个版本的智能合约编写提供代码智能提示功能。

IntelliJ IDE本身提供的代码补全效果是以键入字符序列时自动弹出的下拉列表的形式展现的。插件需要在开发者键入字符序列的同时,实时地展示代码智能提示,将GPT-2模型预测的Top-k选项加入下拉清单中作为备选提示,开发者可以根据编码目的选择最合适的选项。

插件需要基于IntelliJ平台提供的DevKit SDK开发包进行开发,在不影响IDE性能的前提下,将代码智能提示功能与IDE原有的代码补全效果相融合。

插件针对Hyperledger Fabric智能合约提供的代码智能提示功能可以降低智能合约的开发门槛,提升智能合约的开发效率,与此同时,让开发者将注意力集中在业务逻辑的实现上。

\subsection{Fabric智能合约模拟调用模块}

Hyperledger Fabric在网络搭建层面既支持单机上的单个或多个节点,也支持多机多节点部署。客户端应用程序在调用智能合约接口以对区块链账本进行增删改查等逻辑行为之前,至少需要在本地成功搭建一个单节点的Fabric网络。
然而,Hyperledger Fabric的搭建过程较为复杂,而且在Fabric升级到2.0版本后,其平台搭建过程与1.0版本相比产生了较大的区别,智能合约的生命周期也发生了改动。
Hyperledger Fabric智能合约部署流程如图~\ref{fig:3.2}所示。

\begin{figure}[htb]
  \centering
  \includegraphics[width=0.9\linewidth]{figure/requirement/智能合约部署流程.png}
  \caption{智能合约部署流程}\label{fig:3.2}
\end{figure}

由于区块链网络的搭建繁杂耗时,且仅针对智能合约的开发而言不是必要条件,因此插件需要提供一个在无需搭建区块链网络的前提下就能对智能合约进行接口模拟调用的入口。

在实际开发环境中,由于开发者缺少一个针对Hyperledger Fabric智能合约的开发工具,因此在网络中部署的智能合约漏洞频发,导致其无法在Peer节点被执行。当智能合约执行报错时,开发者需要搭建并配置Dev模式进行智能合约调试,每一次修改智能合约后,仍然需要重新启动整个Fabric网络,创建通道,将参与交易的Peer节点加入通道,编译、打包、安装并实例化智能合约,开发者需要经历一系列复杂且冗余的操作后,才能重新尝试执行智能合约。

安全性是区块链技术最关注的因素之一,而网络整体的安全程度很大程度上取决于网络内的节点数量,在多节点的网络内进行智能合约的反复调试和反复部署更加费时费力。

基于IntelliJ平台开发的IDE没有针对Hyperledger Fabric智能合约调试或者测试提供相关的功能,因此,本插件作为一个Hyperledger Fabric智能合约的链下开发工具,需要为开发者提供可以进行智能合约模拟调用的工具窗口,并对测试结果进行展示。

智能合约开发者可以利用本插件的模拟调用及测试功能,在将智能合约发布到区块链网络节点之前就确保其实现和业务逻辑的正确性,既提升了代码质量,又能帮助开发人员节省大量时间,提升开发效率。

\section{非功能性需求分析}

本论文设计并实现的插件需要嵌入到IntelliJ平台的IDE中,所以对其非功能性需求较高。
插件性能需要满足一定标准,在不影响IDE其他功能正常使用的同时,改善开发者使用插件的体验感。
插件的性能、稳定性等重要的非功能性衡量指标如表~\ref{table:requirement}所示:

\begin{table}[htb]\scriptsize
\centering
\caption{非功能性需求列表}
\vspace{2mm}
% l - left, r - right, c - center. | means one vertical line
\begin{tabular}{ccc}
\toprule
\textbf{非功能性需求}&\textbf{目标}&\textbf{优先级}\\
\midrule
\textbf{可扩展性}&支持IntelliJ平台扩展点&高\\ \hline
\textbf{可用性}&良好的界面交互设计&高\\ \hline
\textbf{稳定性}&99.99\%的场景稳定运作&高\\ \hline
\multirow{2}*{\textbf{性能}}&对接口的调用的平均响应时间小于50ms;&\multirow{2}*{高}\\
~&智能合约的代码实时提示展示平均在30ms以内&~\\ \hline
\textbf{可适配性}&支持嵌入多种基于IntelliJ平台开发的IDE&中\\ \hline
\textbf{可部署性}&支持本地构建启动和仓库发布&中\\ \hline
\textbf{可重用性}&具有重用插件扩展点和业务逻辑的能力&中\\ \hline
\textbf{耦合性}&插件与自然语言处理服务解耦&中\\
\bottomrule
\end{tabular}
\label{table:requirement}
\end{table}


插件作为IntelliJ IDE的可插拔式开发辅助工具,需要具备较好的可扩展性。插件的项目结构清晰,且基于IntelliJ平台的DevKit进行开发,可以通过类的继承、配置文件等方式进行功能点或业务逻辑的扩充。

插件需要通过对话框、工具窗口等形式达到良好的可用性,便于开发者的理解和操作,易于上手。

插件需要设计良好的异常处理机制,且稳定地提供NLP服务以维护智能合约代码智能补全的实时效果,具有较高的可靠性。

插件需要进行良好的本地构建和仓库发布,以确保与IntelliJ平台的多种集成开发环境能够兼容适配,便于部署。

IntelliJ平台的插件开发除了可以基于DevKit SDK,还可以基于GitHub Template、Gradle或Maven等构建工具,本插件的功能模块可以快速便捷地迁移到其他方式开发的插件中。

\section{插件的概要设计}

\subsection{插件的总体架构设计}

插件基于IntelliJ平台进行开发,其总体架构设计如图~\ref{fig:3.3}所示:

\begin{figure}[htb]
  \centering
  \includegraphics[width=0.95\linewidth]{figure/design/插件的总体架构设计.png}
  \caption{插件的总体架构设计}\label{fig:3.3}
\end{figure}

面向Hyperledger Fabric智能合约的开发插件主要提供智能合约初始模板生成、代码智能提示、模拟调用及测试等功能。

如上图所示,开发者可以在无需搭建Fabric网络的前提下,利用插件完成智能合约链下开发的整个过程。

IntelliJ平台提供了多种插件搭建方式,本插件的业务功能模块基于DevKit插件进行设计与实现,图中虚线框表示经过实际验证可利用的其他搭建方式,可以将业务功能进行快速迁移。

插件对智能合约开发者的需求的处理流程如下:

\begin{itemize}
  \item 首先,使用插件的智能合约开发者根据业务需求选择合适的智能合约模板,按照FTL语法规则预先定义好的模板文件采用了不同的智能合约开发报,插件将利用FreeMarker模板引擎加载指定文件,并按照模板为用户创建一份初始的智能合约。
  \item 智能合约开发者在初始智能合约的基础上进行业务逻辑填充,在编写代码的同时,插件将提供实时的代码补全提示。系统通过搭建Flask Web服务器以实现对GPT-2模型加载、模型预测的支持,插件将与NLP深度学习的依赖关系进行解耦,通过HTTP请求的方式将单词序列转发给NLP服务并获取模型输出结果。插件在IntelliJ IDE原有的代码补全列表中添加了GPT-2模型预测的代码补全项。
  \item 最后,开发者完成代码编写后,可以打开插件提供的智能合约工具窗口。该窗口基于IntelliJ平台提供的用户接口组件Tool Windows 实现,是开发者进行智能合约模拟调用和测试的入口。开发者可以使用工具窗口获取智能合约及其接口列表,根据业务需求填写接口参数。对于Node.js智能合约进行模拟调用和测试的底层实现使用Hyperledger Fabric提供的MockStub类,同时选择了对JavaScript代码测试提供良好支持的Mocha框架作为辅助。智能合约接口的模拟调用结果反馈在插件自定义的控制台中。
\end{itemize}

\subsection{插件的模块划分设计}

本章从面向Hyperledger Fabric的智能合约开发插件的需求分析中提取了Fabric智能合约初始模板生成、代码智能提示、模拟调用及测试三个关键功能模块:

\begin{itemize}
  \item Fabric智能合约初始模板生成模块,主要负责加载不同FreeMarker模板,创建初始智能合约。
  \item Fabric智能合约代码智能提示模块,主要负责扩展IntelliJ IDE的代码补全功能,增加对Fabric智能合约代码补全的效果。
  \item Fabric智能合约模拟调用及测试模块,主要负责提供工具窗口,查询智能合约及其接口,模拟调用智能合约,在自定义控制台展示测试结果,测试报告的格式通过Mocha设置。
\end{itemize}

每个功能模块中分别包含若干系统模块,系统模块是插件内部开发的关键模块,对用户不可见。
功能模块由实线方框表示,系统模块由虚线方框表示,如图~\ref{fig:3.4}所示:

\begin{figure}[htb]
  \centering
  \includegraphics[width=\linewidth]{figure/design/功能模块图.png}
  \caption{插件的功能及系统模块图}\label{fig:3.4}
\end{figure}

下面将结合UML建模方法对以上三大功能模块的详细设计进行阐述。

\section{插件的详细设计}

\subsection{Fabric智能合约初始模板生成模块设计}

Fabric智能合约初始模板生成模块用于支持初始智能合约的创建,帮助开发者降低智能合约的编写门槛。该模块依赖FreeMarker模板技术定义FTL模板并利用模板引擎进行模板加载;依赖Java IO操作文件读写;依赖IntelliJ平台的Action系统提供用户交互接口。

生成Fabric智能合约初始模板的流程图如图~\ref{fig:3.5}所示。

\begin{figure}[htb]
  \centering
  \includegraphics[width=0.95\linewidth]{figure/design/创建流程.png}
  \caption{生成Fabric智能合约初始模板流程图}\label{fig:3.5}
\end{figure}

IntelliJ平台的用户接口组件Messages支持插件在IDE中弹出各种对话窗口,包括信息提示、警告提示、错误提示等。
插件可以在用户操作出现异常的情况下,给出对用户友好的错误提示信息。
如果用户在当前工作空间未创建任何项目的情况下尝试生成智能合约,插件则会弹出错误提示对话框,提醒用户需要先创建一个项目。
同时,插件通过Messages对话框的形式支持智能合约开发者设置智能合约名称并选择开发方式。

插件基于FreeMarker模板技术生成智能合约文件,FreeMarker模板技术的工作原理示意图如图~\ref{fig:3.6}所示,其引擎执行流程如图~\ref{fig:3.7}所示。

\begin{figure}[htb]
  \centering
  \includegraphics[width=5in]{figure/design/FreeMarker原理.png}
  \caption{FreeMarker工作原理}\label{fig:3.6}
\end{figure}

\begin{figure}[htb]
  \centering
  \includegraphics[width=\linewidth]{figure/design/引擎执行流程.png}
  \caption{FreeMarker引擎执行流程}\label{fig:3.7}
\end{figure}

FreeMarker模板引擎会根据开发者的选择加载指定路径下的FTL模板,并使用Java IO处理文件的创建、读写等操作,最终在符合Hyperledger Fabric开发规范的路径下创建一份初始的智能合约,将FreeMarker预先定义好的模板内容写入智能合约文件。

插件从用户操作体验的角度考虑,为智能合约开发者提供了多种创建智能合约的入口,如表~\ref{table:generate}所示:

\begin{table}[htb]\scriptsize
\centering
\caption{创建智能合约的入口}
\vspace{2mm}
% l - left, r - right, c - center. | means one vertical line
\begin{tabular}{ccc}
\toprule
\textbf{入口}&\textbf{具体操作}&\textbf{快捷键}\\
\midrule
\textbf{IntelliJ IDE编辑器窗口}&右键并在下拉列表中选中Generate&Alt+Insert\\ \hline
\textbf{IntelliJ IDE Tools菜单栏}&展开列表并选中Generate Initial Chaincode&无\\ \hline
\multirow{2}*{\textbf{IntelliJ IDE Help菜单栏}}&展开列表并选中Find Action,&\multirow{2}*{Ctrl+Shift+A}\\
~&在弹出的搜索窗口中输入关键字查找&~\\ \hline
\textbf{自定义快捷键}&同时按下键盘上的Ctrl、Shift、G&Ctrl+Shift+G\\
\bottomrule
\end{tabular}
\label{table:generate}
\end{table}

Fabric智能合约初始模板生成模块类图如图~\ref{fig:3.8}所示。
创建智能合约的入口底层依赖IntelliJ平台的Action系统,需要继承AnAction作为父类,调用父类的构造函数定义入口按钮的名称,重写父类方法~\texttt{actionPerformed()}实现插件业务逻辑,其中AnActionEvent作为接口入参是IntelliJ项目元素的提供者。生成器类Generator是该模块的核心类,负责将数据模型融入FTL模板中。
插件使用TemplateUtil工具类进行FreeMarker模板的配置、加载以及输出。

\begin{figure}[htb]
  \centering
  \includegraphics[width=\linewidth]{figure/design/生成器类图.png}
  \caption{Fabric智能合约初始模板生成模块类图}\label{fig:3.8}
\end{figure}

下文以时序图的形式展示了Fabric智能合约初始模板生成模块的动态模型设计,由此确定了对象之间的事件发生顺序,时序图如图~\ref{fig:3.9}所示。

\begin{figure}[htb]
  \centering
  \includegraphics[width=\linewidth]{figure/design/生成器时序图.png}
  \caption{Fabric智能合约初始模板生成模块时序图}\label{fig:3.9}
\end{figure}

\subsection{Fabric智能合约代码智能提示模块设计}

Fabric智能合约代码智能提示模块用于支持智能合约的实时代码自动补全功能,增强IntelliJ IDE对于编写Fabric智能合约的代码提示效果。

IntelliJ平台提供了多种方式支持插件实现自定义的代码自动补全功能,包括Completion Contributor、Reference-Based Completion以及Live Templates等。
以上方式的区别如表~\ref{table:completionWay}所示。

\begin{table}[htb]\scriptsize
\centering
\caption{代码补全方式列表}
\vspace{2mm}
% l - left, r - right, c - center. | means one vertical line
\begin{tabular}{ccc}
\toprule
\textbf{方式}&\textbf{实现方式概述}&\textbf{适用场景}\\
\midrule
\multirow{2}*{\textbf{Completion Contributor}}&扩展IntelliJ平台提供的CompletionContributor,&\multirow{2}*{针对语言特性}\\
~&填充结果集,注册扩展点&~\\ \hline
\multirow{2}*{\textbf{Reference-Based Completion}}&扩展IntelliJ平台提供的PsiReferenceContributor,&\multirow{2}*{针对自定义语言的属性}\\
~&为自定义语言提供其他语言元素的提示&~\\ \hline
\multirow{2}*{\textbf{Live Templates}}&实现TemplateContextType,&\multirow{2}*{针对简单且固定的需求}\\
~&创建并导出模板,注册扩展点&~\\ 
\bottomrule
\end{tabular}
\label{table:completionWay}
\end{table}

Live Templates可以支持开发者根据编码习惯自定义一些代码片段,减少编写代码的重复性工作。
Live Templates模板创建示例如图~\ref{fig:3.10}所示。

\begin{figure}[htb]
  \centering
  \includegraphics[width=\linewidth]{figure/design/LiveTemplates.png}
  \caption{Live Templates示例}\label{fig:3.10}
\end{figure}

其中Abbreviation设置的文本将作为选项添加到补全列表结果集中,模板内容中\$NAME\$表示输出结果的文本会变红,提示开发者修改代码内容;\$END\$表示创建模板后光标停留的位置。

Hyperledger Fabric官方提供了Node.js智能合约实例,通过观察和研究发现,智能合约具有一些语法特性、编写规律和规范。
虽然模块可以基于Live Templates或Referenced-Based Completion简单地实现Fabric智能合约的自动补全,但是这两种方式只能提供固定模式的代码补全效果,并且依赖人工对智能合约特性的观察与总结,灵活性较差。

因此,插件从需求、灵活度、准确度、性能等方面进行深度考量,确定了此模块的最终设计思路。
以IntelliJ IDE原有的代码补全作为扩展点,选择基于Completion Contributor和自然语言处理模型实现对Fabric智能合约代码提示的功能。

此功能模块底层采用自然语言处理技术针对Fabric智能合约进行深度学习,为智能合约开发者提供高性能、高准确度的编码提示,帮助智能合约开发者降低编码难度,提升开发效率,与此同时,提高智能合约代码质量。

代码智能补全从自然语言处理的角度考虑属于文本生成类型的研究问题。近年来,在文本生成领域涌现了多种预测模型,从2017年Google提出的Transformer模型,到2018年Google和OpenAI分别推出的BERT和GPT模型,以及2019年OpenAI升级后的GPT-2模型,这些生成效果较好的模型都借鉴了自注意力机制。

通过研究和对比,插件采用了GPT-2模型作为Fabric智能合约代码智能提示模块的预测模型,主要有以下几点设计理由:

\begin{itemize}
  \item 参数规模庞大。OpenAI提供了四种参数规模不同的GPT-2,其中最大的Extra Large版本参数规模扩大到15亿,以此得到了更大的模型容量,可以处理更多的训练数据。
  \item 预训练数据体量庞大。GPT-2 的数据集WebText容纳了800万网页,文本集合接近40GB~\cite{DBLP:conf/inlg/KoL20},覆盖各种主题,通用性极强。不仅如此,OpenAI还对这些网页进行了高质量的过滤操作。
  \item 模型结构改进。GPT-2底层基于Transformer模型的解码器,在沿用了Transformer自注意力机制的基础之上,设计了新的Masked Self-Attention机制,解码器模块的堆叠数量翻倍,扩大了字典,增加了输入序列长度。相比传统的基于CNN、RNN的语言模型,GPT-2在特征提取、计算资源利用率、预测性能上都有较大的改进。
\end{itemize}

综上,采用GPT-2模型生成的文本具有较好的连贯性,预测的准确率较高,适合应用于代码自动补全领域。

插件需要实现对于IntelliJ平台的扩展点,即自定义的Completion Contributor,在该扩展点中获取当前编辑器中正在键入的文本序列,将其通过HTTP请求的方式发送给Flask框架。
Flask将把接收到的请求路由到对应的接口,模型将服务器接收到的请求数据作为输入进行模型预测,将预测的结果列表作为输出返回给插件。
最后,插件将预测选项添加到原有的代码补全列表中。
Fabric智能合约代码智能提示模块的流程图如图~\ref{fig:3.11}所示。

\begin{figure}[htb]
  \centering
  \includegraphics[width=5in]{figure/design/代码智能提示流程图.png}
  \caption{Fabric智能合约代码智能提示模块流程图}\label{fig:3.11}
\end{figure}

由需求分析可知,此模块需要针对Fabric智能合约给出代码智能提示。因此,模块的数据集应该选取Fabric Node.js智能合约,由于智能合约是一段代码程序,其中可能包含大量注释,而这些注释内容并不是代码补全的目标,如果保留还会影响模型学习的上下文,所以需要对采集的数据进行预处理,删除Node.js智能合约中各种形式的注释。

清洗之后的数据集还需要经历分词编码才能放入GPT-2模型中进行无监督学习。由于自然语言处理的训练过程对硬件要求较高,因此本论文采用GPU进行辅助,模型训练过程中需要不断调整参数,也可以通过可视化历史数据的方式对比观察,提升模型效果。

插件将应用训练好的GPT-2模型支持代码自动补全功能,该功能需要提供实时效果且不能影响IntelliJ IDE的正常运行,因此对自然语言处理技术所提供的服务速度要求较高。

由于分词器和GPT-2模型每次加载都会耗费较长时间,而实际预测速度较快,因此插件采用Flask Web服务器提供自然语言处理的服务.
这种方式可以将模型进行预加载从而节省大量时间,同时也将IntelliJ插件与GTP-2模型做了良好的解耦,解耦示意图如图~\ref{fig:3.12}所示。

\begin{figure}[htb]
  \centering
  \includegraphics[width=0.95\linewidth]{figure/design/解耦示意图.png}
  \caption{插件与模型解耦示意图}\label{fig:3.12}
\end{figure}

如上图所示,Flask服务器提供对外暴露的接口,接口描述如表~\ref{table:apiDesc}所示。

\begin{table}[htb]\scriptsize
\centering
\caption{接口描述}
\vspace{2mm}
% l - left, r - right, c - center. | means one vertical line
\begin{tabular}{ccccc}
\toprule
\textbf{调用者}&\textbf{请求方式}&\textbf{接口调用地址}&\textbf{入参字段类型}&\textbf{入参字段名}\\
\midrule
Completion Contributor&POST&http://{ip:port}/plugin/completor&String&dataSequence\\
\bottomrule
\end{tabular}
\label{table:apiDesc}
\end{table}

下文以时序图的形式展示了Fabric智能合约代码智能提示模块的动态模型设计,以此明确该模块中的对象关系和模块处理整个事件的先后顺序,时序图如图~\ref{fig:3.13}所示。

\begin{figure}[htb]
  \centering
  \includegraphics[width=\linewidth]{figure/design/代码提示时序图.png}
  \caption{Fabric智能合约代码智能提示模块时序图}\label{fig:3.13}
\end{figure}

\subsection{Fabric智能合约模拟调用模块设计}

Fabric智能合约模拟调用模块用于支持智能合约开发者进行接口模拟调用、调试及测试,便于开发者在将智能合约发布到区块链网络节点之前,确保其代码编写的正确性。

该模块在用户接口层面的实现依赖IntelliJ平台提供的用户接口组件Tool Windows,插件的图形用户界面布局的实现依赖Java提供的GUI开发工具包Swing。

IntelliJ IDE的整个窗口界面包含了多个工具窗口,分别在IDE界面的左侧、底部和右侧。
这些工具窗口相当于IDE的子窗口,可以独立地支持不同的功能需求,例如展示项目结构、集成其他工具、运行并调试应用程序等,工具窗口示意图如图~\ref{fig:3.14}所示。

\begin{figure}[htb]
  \centering
  \includegraphics[width=\linewidth]{figure/design/工具窗口示意图.png}
  \caption{IntelliJ IDE工具窗口示意图}\label{fig:3.14}
\end{figure}

模块自定义的工具窗口为智能合约开发者提供了与插件功能交互的入口,其内部共分为三个区域,分别是工具栏Tool Bar、智能合约接口面板以及模拟调用控制台。该工具窗口支持智能合约开发者在编辑窗口完成智能合约的编码工作后,进行接口的模拟调用及测试。

Tool Bar需要向用户提供若干按钮,以支持查询智能合约及其接口信息,调用智能合约接口,重置智能合约接口面板等操作。

智能合约接口面板以表格的形式展示了当前工作空间中的智能合约及其接口列表,并在表格中设置了文本框区域,支持智能合约开发者根据业务需求填写接口参数。

模拟调用控制台是模块工具窗口中的一个自定义控制台,作为反馈智能合约正确性的区域,展示了接口测试的日志及结果。

在Hyperledger Fabric网络中,SDK、CLI等客户端应用程序会调用运行在Peer节点的智能合约,而智能合约的业务逻辑代码可以利用ChaincodeStub类提供的接口,从而实现对于存储在区块中账本数据的操作,数据更新流程如图~\ref{fig:3.15}所示。

账本中的每一条数据是以世界状态键值对的形式存储的,目前Hyperledger Fabric提供了两种状态数据库,分别是CouchDB和LevelDB~\cite{DBLP:conf/hpcc/Dyreson16},两者都支持键值状态存储。

\begin{figure}[htb]
  \centering
  \includegraphics[width=\linewidth]{figure/design/账本数据更新流程.png}
  \caption{账本数据更新流程图}\label{fig:3.15}
\end{figure}

由需求分析可知,插件需要在无需搭建Hyperledger Fabric网络环境的前提下,实现链下的Node.js智能合约接口调用。因此,本模块底层基于Hyperledger Fabric提供的一种模拟桩ChaincodeMockStub实现。之所以能够利用ChaincodeMockStub类模拟智能合约接口的调用,是因为该类的内部维护了一个类似key-value状态数据库的Map<String,[]byte>,对该类接口的调用会作用于内存中的Map。

同时,该模块的实现需要搭建Node.js环境提供支持,依赖FreeMarker模板技术结合当前主流且简单灵活的JavaScript测试框架Mocha共同完成。模块需要查找并提取智能合约的接口和参数,将提取内容作为FreeMarker模板的数据模型,引擎将加载预先定义好的FTL模板并生成接口调用及测试脚本,模板内容需要按照FTL语法规则进行编写。插件需要对脚本调用结果进行处理,并展示到工具窗口自定义的控制台中。

模块需要打印智能合约模拟调用结果,包括接口调用成功个数、失败个数、日志、测试覆盖率等。智能合约开发者可以根据控制台打印结果进行代码调试,在智能合约相应代码行设置断点,并以debug模式运行插件。

上文阐述了Fabric智能合约模拟调用模块的整体设计思路,模块核心功能的程序执行流程如图~\ref{fig:3.16}所示。

\begin{figure}[htb]
  \centering
  \includegraphics[width=5in]{figure/design/模拟调用模块流程图.png}
  \caption{模拟调用程序执行流程图}\label{fig:3.16}
\end{figure}

下文以类图的形式描述了Fabric智能合约模拟调用模块的静态结构,展示了该模块的类、对象、属性以及操作。该模块主要围绕的核心类是MockerToolWindow以及ChaincodeApi。
Fabric智能合约模拟调用模块类图如图~\ref{fig:3.17}所示。

\begin{figure}[htb]
  \centering
  \includegraphics[width=\linewidth]{figure/design/模拟调用模块类图.png}
  \caption{Fabric智能合约模拟调用模块类图}\label{fig:3.17}
\end{figure}

MockerToolWindow作为插件自定义的工具窗口,其设计与实现的内容将被工具工厂加载,工具窗口内部的智能合约面板区域以表格形式呈现,需要实现自定义的表格渲染器MyTableCellRender以绘制表格。

Fabric智能合约模拟调用模块设计并抽象了智能合约接口类ChaincodeApi,将智能合约中的每个接口及其相关信息作为List的一项,最终添加到FreeMarker模板的数据模型Map中。

本插件不仅可以帮助智能合约开发者验证代码编写的语法正确性,还通过Fabric智能合约模拟调用模块保证了智能合约中业务逻辑的连贯性和一致性,同时考虑了用户使用插件的体验感。

\section{本章小结}

本章从功能性需求和非功能性需求的角度对面向Hyperledger Fabric智能合约的开发插件进行细致的分析。
通过用例图展示了插件的功能性需求,并分别对智能合约初始模板生成、代码智能提示、模拟调用及测试等功能进行需求分析;
非功能性需求从插件的性能、可适配性、可用性等方面考虑并进行阐述。
同时,本章基于插件的需求分析进行系统设计,包括概要设计和详细设计。
概要设计部分展示了面向Hyperledger Fabric智能合约的开发插件的总体架构以及功能模块的划分设计。
详细设计部分阐述了Fabric智能合约初始模板生成、代码智能提示、模拟调用及测试三个功能模块的设计思路,同时结合静态、动态模型等方式进行了细致的阐述。

\chapter{插件的实现}

本章主要介绍面向Hyperledger Fabric的智能合约开发插件的实现细节。
基于概要设计和详细设计,阐述Fabric智能合约初始模板生成、代码智能提示、模拟调用及测试三个功能模块的技术实现。
同时,本章将对部分功能的实现结果进行展示。

\section{Fabric智能合约初始模板生成模块实现}

\subsection{生成器用户接口实现}

由需求分析和系统设计可知,此模块旨在为智能合约开发者生成一份初始的Fabric智能合约。
因此,插件本质上需要实现一个代码生成器,该代码生成器的目标对象是Fabric智能合约。

插件基于IntelliJ平台提供的DevKit SDK进行开发,其中包含的Action系统可以支持生成器实现用户接口层,代码结构如图~\ref{fig:5.1}所示。

\begin{figure}[htb]
  \centering
  \includegraphics[width=0.65\linewidth]{figure/implement/生成器代码结构.pdf}
  \caption{生成器用户接口层代码结构}\label{fig:5.1}
\end{figure}

Generator作为自定义Action需要继承AnAction作为父类,调用父类构造函数以定义入口按钮的名称,并重写父类方法\texttt{actionPerformed()}实现代码生成器的业务逻辑。
在业务逻辑中通过调用参数AnActionEvent的\texttt{getProject()}方法可以获取当前工作空间的项目。

为了使得自定义的Action在IntelliJ IDE中生效,需要在配置文件中注册该Action,配置代码如图~\ref{fig:5.2}所示。

\begin{figure}[htb]
  \centering
  \includegraphics[width=\linewidth]{figure/implement/Action注册实现.pdf}
  \caption{Action注册实现}\label{fig:5.2}
\end{figure}

如上图所示,为了使插件具有良好的用户体验感,系统为智能合约开发者提供了代码生成器的多个入口,包括工具菜单栏、代码行右键的生成按钮、帮助菜单中的搜索框以及快捷键Ctrl+Shift+G。
配置文件中自定义Action的各个配置项描述如表~\ref{table:generateXml}所示。

\begin{table}[htb]\scriptsize
\centering
\caption{Action配置项描述}
\vspace{2mm}
% l - left, r - right, c - center. | means one vertical line
\begin{tabular}{cc}
\toprule
\textbf{属性}&\textbf{描述}\\
\midrule
\multirow{2}*{\textbf{Action ID}}&Action在插件中的特殊标识符,若为一个Action提供多个入口,\\
~&则需要不同ID,例如在ID基础上添加后缀\\ \hline
\multirow{2}*{\textbf{Class}}&插件Action的实现类,需要继承AnAction,\\
~&可以被多个Action重用\\ \hline
\textbf{Text}&插件Action在菜单中显示的文本\\ \hline
\textbf{Description}&插件Action设置的简要描述信息\\ \hline
\textbf{Icon}&插件Action图标路径\\ \hline
\multirow{3}*{\textbf{Add-to-group}}&Group-id标识着插件Action加入的Action组,\\
~&指定了Action在IDE中的位置;Anchor代表Action锚点,\\ 
~&设置在Action Group中该Action相对其他Action的位置\\ \hline
\textbf{Keyboard-shortcut}&定义了插件Action的快捷键\\
\bottomrule
\end{tabular}
\label{table:generateXml}
\end{table}

该模块依赖IntelliJ平台的用户接口组件Messages实现了良好的用户交互。
Messages提供了多种IntelliJ IDE弹窗,包括带有文本框的输入弹窗、带有组合框的选择弹窗以及多种信息弹窗。

插件根据不同需求场景采用了对应的弹窗进行用户交互。
开发者当前的工作空间存在项目是生成智能合约初始文件的前提,因此,如果插件检测到没有任何项目存在,就会弹出错误窗口,提醒开发者必须创建一个新项目;
插件通过输入弹窗让开发者可以设置智能合约的名称;
插件通过选择弹窗让开发者设置智能合约的开发模式;
最后,插件通过信息弹窗通知开发者智能合约初始模板的创建状态。
代码片段如图~\ref{fig:5.3}所示。

\begin{figure}[htb]
  \centering
  \includegraphics[width=0.8\linewidth]{figure/implement/用户交互弹窗实现.pdf}
  \caption{用户交互弹窗实现}\label{fig:5.3}
\end{figure}

\subsection{自定义Action实现}

模块在自定义Action内部实现Fabric智能合约生成器的业务逻辑,该过程基于FreeMarker模板技术以及Java IO文件操作。
Fabric智能合约生成器首先通过用户接口层获取开发者设置的智能合约名称以及选择的智能合约开发模式,然后基于获取到的数据构建FTL模板的数据模型(DataModel),FreeMarker引擎加载模板并处理数据模型,最后利用Java IO创建智能合约文件并将模板内容写入文件。
Fabric智能合约生成器的业务逻辑代码实现片段如图~\ref{fig:5.4}所示。

\begin{figure}[htb]
  \centering
  \includegraphics[width=0.8\linewidth]{figure/implement/Fabric智能合约生成器业务逻辑.pdf}
  \caption{Fabric智能合约生成器实现}\label{fig:5.4}
\end{figure}

插件封装FreeMarker模板提供的服务并实现了模板工具类TemplateUtil。
Fabric智能合约生成器将调用TemplateUtil提供的静态方法完成对FreeMarker引擎的配置和模板的加载,处理数据模型并输出,TemplateUtil代码实现片段如图~\ref{fig:5.5}所示。

\begin{figure}[htb]
  \centering
  \includegraphics[width=\linewidth]{figure/implement/工具类TemplateUtil实现.pdf}
  \caption{工具类TemplateUtil实现}\label{fig:5.5}
\end{figure}

FreeMarker模板引擎可以通过内建模板加载器加载指定文件,调用\texttt{setClassForTemplateLoading()}方法设置类文件路径,或调用~\texttt{setDir-} \linebreak ~\texttt{ectioryForTemplateLoading()}方法设置文件的绝对路径,还可以基于Web服务器的路径加载模板。
结合需求分析并考虑到本插件不仅限于本地使用,因此模块采用类路径加载方式,将FTL模板放在系统的resources资源中。

FreeMarker模板引擎的配置类Configuration需要确定引擎的版本从而完成实例化,调用Configuration的\texttt{setTemplateExceptionHandler()}方法设置引擎的异常处理机制。

FreeMarker模板引擎提供了常见场景下采用的几种异常处理机制,如表~\ref{table:exception}所示。
由于插件在自定义Action中需要针对不同d 异常情况做相应的处理,因此设置引擎的异常处理器为RETHROW\_HANDLER,当引擎运行出错时不进行任何处理直接重新抛出错误,将控制权交还给插件本身。

\begin{table}[htb]\scriptsize
\centering
\caption{FreeMarker引擎异常处理机制}
\vspace{2mm}
% l - left, r - right, c - center. | means one vertical line
\begin{tabular}{cc}
\toprule
\textbf{名称}&\textbf{描述}\\
\midrule
\textbf{DEBUG\_HANDLER}&默认机制,控制台打印堆栈信息\\ \hline
\multirow{2}*{\textbf{HTML\_DEBUG\_HANDLER}}&适合使用FreeMarker输出HTML页面时使用,\\
~&支持堆栈信息格式化\\ \hline
\textbf{IGNORE\_HANDLER}&直接忽略所有异常情况\\ \hline
\textbf{RETHROW\_HANDLER}&重新抛出异常,适合Web应用等场景\\
\bottomrule
\end{tabular}
\label{table:exception}
\end{table}

\section{Fabric智能合约代码智能提示模块实现}

插件采用自然语言处理模型GPT-2为此模块提供支持,增强IntelliJ IDE对Fabric Node.js智能合约的代码补全效果。
实现步骤主要包括:数据采集、数据预处理、模型训练、模型应用。
下面将详细展开介绍。

\subsection{数据采集}

基于需求分析,此模块需要针对Fabric Node.js智能合约的编码给出智能提示,那么Fabric Node.js智能合约即为GPT-2模型需要学习并生成的文本数据。
因此,模块选择GithHub开源网站进行数据采集,使用Python语言编写网页爬虫爬取智能合约文件作为模型输入。
网页爬虫的核心是基于URL过滤并锁定数据,从而自动爬取特征范围内的网页。爬虫代码实现片段如图~\ref{fig:5.6}所示。

\begin{figure}[htb]
  \centering
  \includegraphics[width=\linewidth]{figure/implement/爬虫实现.pdf}
  \caption{爬虫实现}\label{fig:5.6}
\end{figure}

Hyperledger Fabric为Node.js智能合约的编写提供了两种开发包,在智能合约中必须引入二者之一。
因此,模块采集的数据特征是代码文件中包含关键字“fabric-shim”或“fabric-contract-api”。
通过关键字搜索的文件结果数量远大于网站规定的可显示数量,因此在URL中采取限制文件大小的方式缩小结果数量,以此抓取所有满足条件的智能合约。
经过对智能合约和GitHub网站的深入分析,得到上图代码需要爬取的URL为“https://github.com/search?l=JavaScript\&p=num\&q=require(fabric-shim)+size\%3A \linebreak >10000\&type=Code”。
Python爬虫最终抓取的Fabric Node.js智能合约总计6151份。

\subsection{数据预处理}

数据预处理分为两个部分,一是数据清洗,二是分词编码。

数据清洗主要针对智能合约文件中的注释。
因为插件的智能提示目标是代码段,所以注释不是模型的学习对象,其存在反而会影响代码的上下文。

代码文件中的注释有两种类型,第一种是以“//”开头的单行注释,第二种是被“/*”和“*/”包裹的多行注释。
数据清洗脚本首先将代码复制到一个临时文件中,接着用指针遍历文件中的字符,识别关键的起始符号以确认是否是注释。
非注释的代码语句都将被重新写入文件中,最后,临时文件会被删除。

对于清理注释的方法,一共设置了七个状态,以便指针运动的过程中能够分辨出当前字符是否在注释中。
首先文档的状态初始设置为INIT状态。
指针遍历文档中每一行的每一个字符,读入字符时判断当前状态:

\begin{itemize}
  \item INIT状态:如果当前字符为"/",文档状态切换到SLASH,以标记遇到了第一个"/",接着遍历下一个字符;如果当前字符为引号,则表明进入了一对引号引起的字符串,此时将状态切换到STR,并将当前字符写入文件;在其他情况下,说明当前不在注释内,直接写入当前字符。
  \item SLASH状态:如果当前字符为"*",表明了一个多行块状注释的起始,此时将状态切换为BLOCK\_COMMENT;若当前字符为"/",表明了一个单行注释的起始,此时将状态切换为LINE\_COMMENT;若当前字符为其他字符,则将前一位的"/"和当前字符一起写入文件,并将状态切换回INIT。
  \item BLOCK\_COMMENT状态:如果当前字符为"*",则表明可能已遇到块状注释结尾的倒数第二个字符,此时将状态切换为BLOCK\_COMMENT\_DOT;若为其他字符,表明当前还在注释块中,直接进入下一个字符。
  \item BLOCK\_COMMENT\_DOT状态:如果当前字符为"/",表明当前注释块已到结尾,此时状态切换回INIT;若为其他字符,则表明还在注释块中,进入下一个字符。
  \item LINE\_COMMENT状态:由于该行均为注释,只有当前字符为换行符时,才将状态切换为INIT,接着进入下一行。
  \item STR状态:表明当前在字符串中。若当前字符为反斜杠,则将状态切换为STR\_ESCAPE;若当前字符为引号,表明字符串结束,状态切换为INIT;若当前字符为其他字符,状态保持不变。最后写入当前字符。
  \item STR\_ESCAPE状态:将状态切换回STR,接着写入当前字符到文件,并进入下一个字符。
\end{itemize}

模块采用的分词器(Tokenizer)使用UTF-8编码方式读取数据,为了防止Tokenizer加载数据异常,在分词编码之前会对数据的编码格式进行清洗。

Tokenizer利用字节对编码(Byte Pair Encoder,BPE)对文本序列进行分词,代码实现片段如图~\ref{fig:5.7}所示。

\begin{figure}[htb]
  \centering
  \includegraphics[width=0.8\linewidth]{figure/implement/分词编码实现.pdf}
  \caption{分词编码实现}\label{fig:5.7}
\end{figure}

BPE会根据统计数据和词汇表对单词序列进行拆分,得到不同的字母组合,单词序列经过编码后会得到多个ID,模型预测的结果ID会经过解码得到字母组合。
上图代码中~\texttt{bpe\_train()}为训练分词模型的方法,使用定义好的分词模型对Node.js智能合约文件进行分词和编码,并将相关信息保存在文件中。


\subsection{GPT-2模型训练}

由于自然语言模型的学习过程对硬件需求较高,所以选择将模型部署在GPU机器上。
使用Anaconda环境管理工具隔离模型执行所依赖的环境。
安装CUDA和TensorFlow框架支持GPT-2模型的搭建和训练。

首先编写脚本准备输入数据,代码实现片段如图~\ref{fig:5.8}所示。

\begin{figure}[htb]
  \centering
  \includegraphics[width=0.85\linewidth]{figure/implement/准备模型输入数据脚本.pdf}
  \caption{切分数据样本脚本}\label{fig:5.8}
\end{figure}

脚本将每个智能合约文件中的语句连接起来形成一个字符串,其中文件与文件之间加入特殊字符以便标识文件的结尾,Tokenizer将得到的字符串按照词汇表进行分词编码。
然后将已成为编码序列的数据切分成多个块,块中容纳100个词,对于每一个块,将数据根据上下文分为输入集和标签集。
将输入集划分为训练集、验证集和测试集,其中训练集和测试集按照8:2的比例切分,而训练集中又按照8:2的比例分割成最终的训练集和验证集。

TensorFlow会加载准备好的数据集,并根据设定的buffer size和batch size将数据打乱。

GPT-2模型底层基于Transformer解码器,采用transformers包提供的GPT2Config、TFGPT2LMHeadModel和GPT2Tokenizer包实现模型训练过程,代码实现片段如图~\ref{fig:5.9}所示。

\begin{figure}[htb]
  \centering
  \includegraphics[width=0.85\linewidth]{figure/implement/模型训练脚本.pdf}
  \caption{模型训练脚本}\label{fig:5.9}
\end{figure}

脚本首先加载已保存的分词模型,使用分词模型的词汇表作为配置注入GPT-2并形成适应数据的模型。
模型训练的优化函数使用了~\texttt{Adam()};损失函数使用的是稀疏类别交叉熵。
脚本设置了回调函数以帮助模型在性能无法提升时能够提前终止训练,回调函数监控的指标为验证集的损失(Loss)。

同时,对训练过程中训练集和验证集的Loss进行记录,并基于MatPlotLib进行可视化,便于观察和评估模型训练效果。
如图~\ref{fig:5.10}所示。

\begin{figure}[htb]
  \centering
  \includegraphics[width=0.6\linewidth]{figure/implement/Loss.png}
  \caption{训练集和验证机Loss可视化}\label{fig:5.10}
\end{figure}

数据可视化后可以清晰地观察到,模型在验证集中的Loss在第十轮训练后已经趋于平缓,且根据回调函数可知,Loss在第二十一轮开始几乎不再下降,所以认为模型训练可以终止。

\subsection{基于GPT-2模型的代码智能提示的实现}

上文介绍了代码智能提示模块的数据采集、数据预处理以及模型训练等步骤的实现细节。
训练好的GPT-2模型共计451MB,包含了模型结构、分词器、词汇表等内容,如图~\ref{fig:5.11}所示。

\begin{figure}[htb]
  \centering
  \includegraphics[width=0.4\linewidth]{figure/implement/GPT-2目录.png}
  \caption{GPT-2模型结构}\label{fig:5.11}
\end{figure}

由需求分析和系统设计可知,代码智能提示模块采用了Flask Web框架提供自然语言处理服务。
Flask框架需要进行环境配置,使用pip install命令安装flask及相关依赖包。

模块中编写的Flask应用程序依赖OpenAI官方提供的两个开发包GPT2To- \linebreak kenizer和TFGPT2LMHeadModel,调用~\texttt{from\_pretrained()}方法加载分词器和GPT-2模型。
Flask服务器提供RESTFUL接口以接收HTTP请求传送的数据,调用分词器的~\texttt{encode()}方法对输入数据进行编码,调用模型的~\texttt{generate()}方法得到模型预测结果,Flask应用程序代码实现片段如图~\ref{fig:5.12}所示。

\begin{figure}[htb]
  \centering
  \includegraphics[width=0.75\linewidth]{figure/implement/Flask应用程序实现.pdf}
  \caption{Flask应用程序实现}\label{fig:5.12}
\end{figure}

启动Flask服务器预先对Tokenizer和GPT-2模型进行加载,为插件的实时代码补全功能提供高稳定和高性能的支持,同时使得插件开发和自然语言处理服务实现良好解耦,弱化彼此牵制的影响。

此模块基于IntelliJ平台提供的CompletionContributor接口实现了代码自动补全扩展点,并在plugin.xml中注册该扩展点使其生效,配置文件代码片段如图~\ref{fig:5.13}所示。
由于此插件面向Node.js智能合约的开发,因此,设置该扩展点只在开发者编写JavaScript类型文件时生效。

\begin{figure}[htb]
  \centering
  \includegraphics[width=\linewidth]{figure/implement/代码自动补全扩展点注册实现.pdf}
  \caption{代码自动补全扩展点注册实现}\label{fig:5.13}
\end{figure}

插件在completion.contributor扩展点中制定了自定义的代码自动补全机制GPTCompletionContributor,该类需要实现CompletionContributor接口,重写~\texttt{fillCompletionVariants()}方法,下文将阐述该方法的业务逻辑。

首先,在方法中利用CompletionParameters获取当前编辑窗口键入序列,代码实现片段如图~\ref{fig:5.14}所示。

\begin{figure}[htb]
  \centering
  \includegraphics[width=0.85\linewidth]{figure/implement/获取键入文本序列实现.pdf}
  \caption{获取键入文本序列实现}\label{fig:5.14}
\end{figure}

IntelliJ平台提供了程序结构接口(Program Structure Interface,PSI),插件利用PSI元素探索了IntelliJ平台的源代码结构,执行代码分析。
如上文代码片段所示,CompletionParameters封装了多个PsiElement属性。
调用~\texttt{getOriginalPosition()}方法返回的original对象是当前光标所在代码行的PsiElement,调用~\texttt{parameters.getOffset()}获取当前代码行内输入字符的偏移量,调用~\texttt{parameters.getPosition().getTextRange().get-} \linebreak ~\texttt{StartOffset()}获取当前代码行内输入文本序列前缀的起始偏移量,两者之差textLength即为当前代码行起始下标到光标之间的长度,从original对象中截取长度为textLength的字符串得到当前编辑窗口键入的文本序列。

获取模型输入后,在~\texttt{fillCompletionVariants()}方法中调用Flask服务器的对外暴露接口发送文本序列,获取GPT-2模型预测的结果集,最后将结果集添加到代码自动补全列表中,代码实现片段如图~\ref{fig:5.15}所示。

\begin{figure}[htb]
  \centering
  \includegraphics[width=0.7\linewidth]{figure/implement/代码智能提示实现.pdf}
  \caption{代码智能提示实现}\label{fig:5.15}
\end{figure}

IntelliJ平台将代码自动补全列表中的每个条目封装成LookupElement类,使用LookupElementBuilder创建新的元素,最后调用~\texttt{addElement()}方法将元素加入CompletionResultSet中。

\section{Fabric智能合约模拟调用及测试模块实现}

Fabric智能合约模拟调用模块的底层基于FreeMarker模板技术,结合Hyperledger Fabric提供的模拟桩ChaincodeMockStub和JavaScript测试框架Mocha,实现了智能合约接口的模拟调用、调试和测试功能。
同时,插件也为智能合约开发者提供了友好的用户交互接口。

\subsection{模拟调用及测试的实现}

智能合约在实际的Hyperledger Fabric网络中的执行过程如图~\ref{fig:5.16}所示。
插件依据智能合约链上的执行原理实现了链下对智能合约接口的模拟调用。

\begin{figure}[htb]
  \centering
  \includegraphics[width=0.7\linewidth]{figure/implement/智能合约执行过程.png}
  \caption{智能合约执行过程}\label{fig:5.16}
\end{figure}

插件搭建了Node.js环境支持Mocha框架的使用,该框架作为代码质量保证的工具,对JavaScript代码的测试提供了强大的支持,功能丰富且灵活,并且可以提供准确和简洁的测试报告。

在Fabric智能合约模拟调用模块中,FreeMarker模板引擎加载的FTL模板内容是基于Mocha框架编写的智能合约接口测试脚本,脚本利用ChaincodeMockStub调用智能合约接口并获取返回值,同时使用其内置的键值对数据结构模拟Hyperledger Fabric网络的状态数据库。
FTL模板代码实现片段如图~\ref{fig:5.17}所示。

\begin{figure}[htb]
  \centering
  \includegraphics[width=\linewidth]{figure/implement/FTL模板实现.pdf}
  \caption{FTL模板实现}\label{fig:5.17}
\end{figure}

如上图所示,测试脚本中的每一个describe和it代码块都可以独立执行,describe代码块是脚本中的测试套件,在该代码块内的测试具有相关性,可以为describe测试套件设置名称和执行函数;
it代码块作为测试的最小单元,是脚本中的测试用例,与describe测试套件类似,也可以为it测试用例设置名称和执行函数。

Mocha还提供了对测试用例使用钩子的支持,如表~\ref{table:mocha}所示。
模块需要实现对Fabric智能合约接口的模拟调用,因此执行测试脚本describe测试套件的所有it测试用例之前,采用了Mocha框架的钩子之一~\texttt{before()}。
在~\texttt{before()}的执行函数中利用ChaincodeMockStub为Fabric智能合约的ChaincodeStubInterface创建模拟桩,并对智能合约进行初始化。

\begin{table}[htb]\scriptsize
\centering
\caption{Mocha钩子执行时间}
\vspace{2mm}
% l - left, r - right, c - center. | means one vertical line
\begin{tabular}{cc}
\toprule
\textbf{函数}&\textbf{执行时间}\\
\midrule
\textbf{before()}&在describe测试套件内的所有it测试用例之前执行\\ \hline
\textbf{after()}&在describe测试套件内的所有it测试用例之后执行\\ \hline
\textbf{beforeEach()}&在describe测试套件内的每个it测试用例之前执行\\ \hline
\textbf{afterEach()}&在describe测试套件内的每个it测试用例之后执行\\
\bottomrule
\end{tabular}
\label{table:mocha}
\end{table}

插件将智能合约名称、接口封装成数据模型融入FTL模板中,FreeMarker引擎加载模板输出一份智能合约模拟调用及测试脚本。
插件利用Mocha框架执行脚本并获取命令行打印结果,命令行输出分为标准输出和错误输出,为了防止两者之间互相阻塞,模块采用多线程的形式获取脚本执行日志,代码实现片段如图~\ref{fig:5.18}所示。
插件通过设置不同的Mocha命令行执行参数配置脚本的执行方式、控制测试报告的生成形式,执行参数的设置可以写入配置文件。
Mocha框架提供的测试报告中包含了脚本运行花费的时间,执行测试用例成功和失败的个数及其具体执行细节。

\begin{figure}[htb]
  \centering
  \includegraphics[width=0.8\linewidth]{figure/implement/执行脚本实现.pdf}
  \caption{执行脚本实现}\label{fig:5.18}
\end{figure}

\subsection{IntelliJ IDE工具窗口的实现}

Fabric智能合约模拟调用模块为了提高功能易用性,基于IntelliJ平台提供的用户接口组件Tool Windows实现了友好的用户接口层。

模块定义了MockerToolWindowFactory,该类作为工具窗口的工厂实现了ToolWindowFactory接口,在plugin.xml配置文件中注册扩展点toolWindow使得自定义的工具窗口在IntelliJ IDE界面生效,注册工具窗口代码实现片段如图~\ref{fig:5.19}所示。

\begin{figure}[htb]
  \centering
  \includegraphics[width=\linewidth]{figure/implement/注册工具窗口.pdf}
  \caption{注册工具窗口}\label{fig:5.19}
\end{figure}

配置文件中自定义工具窗口的属性id标识了窗口名称。IntelliJ IDE中的左侧、底部及右侧的工具窗口区域都包含主要组和次要组,其中次要组中的工具窗口不能同时打开,配置文件中属性secondary设为true,表示该工具窗口是次要的。
配置文件中的属性anchor设置为right,则该工具窗口在界面右侧区域展开。
属性factoryClass指定了自定义工具窗口的实现工厂。

当智能合约开发者打开工具窗口时,MockerToolWindowFactory重写接口的~\texttt{createToolWindowContent()}方法将被调用,初始化工具窗口UI,该机制保证了工具窗口未打开时并不会占用内存而影响IDE性能。
MockerToolWindowFactory工厂类的代码实现片段如图~\ref{fig:5.20}所示。
方法中调用了ContentFactory的~\texttt{createContent()}方法为工具窗口创建组件内容,然后获取到工具窗口的ContentManager并将内容添加到工具窗口。

\begin{figure}[htb]
  \centering
  \includegraphics[width=\linewidth]{figure/implement/自定义工具窗口工厂类实现.pdf}
  \caption{自定义工具窗口工厂类实现}\label{fig:5.20}
\end{figure}

MockerToolWindow基于Java Swing的开发包实现了智能合约模拟调用工具窗口的GUI。
该窗口由Tool Bar、智能合约接口列表面板、自定义控制台三部分构成。
Tool Bar提供了带有监听器的按钮,开发者点击相应按钮可以查询智能合约及其接口、模拟调用、重置面板等。
插件实现了自定义的TableCellRender渲染器,以此控制智能合约接口列表的格式。
自定义控制台对执行智能合约模拟调用脚本的输出日志和结果进行展示。
工具窗口界面如图~\ref{fig:5.21}所示。

\begin{figure}[htb]
  \centering
  \includegraphics[width=\linewidth]{figure/implement/工具窗口.png}
  \caption{智能合约模拟调用工具窗口}\label{fig:5.21}
\end{figure}

\section{本章小结}

本章基于插件的概要设计和详细设计对各个模块功能进行代码实现和结果展示。结合插件的关键代码分别阐述了Fabric智能合约初始模板生成、代码智能提示、模拟调用三个模块的开发细节。


\chapter{插件的测试}

本章主要介绍面向Hyperledger Fabric的智能合约开发插件的测试工作。首先概述了测试的目标及方式;然后基于模块的功能划分设计了测试用例从而完成功能测试,同时对插件应用的自然语言处理模型做了评估;
最后,基于插件的非功能性需求进行了性能测试。

\section{测试目标及方式}

本文始终依照软件工程方法执行每一个环节,在插件的整个研发过程中,测试工作的跟进具有重要价值。

本章的测试目标主要有以下几点:

\begin{itemize}
    \item 验证插件各模块的功能实现是否满足智能合约开发者的功能性需求。
    \item 保证插件所提供服务具有稳定性且满足性能要求。
    \item 评估插件的质量,为插件的开发提供依据和帮助。
    \item 检查插件的开发是否存在漏洞,并发现插件值得改进和扩展的地方。
\end{itemize}

在确定测试目标后,围绕功能测试和性能测试展开测试工作。由于插件基于IntelliJ平台开发,因此对插件的测试基于IntelliJ平台的测试框架,通过扩展LightCodeInsightFixtureTestCase类实现了插件的轻量级测试。

插件的测试环境分为两部分,一个是本地开发环境,一个是提供GPT-2自然语言处理模型服务的Google Cloud环境。
两个环境及其配置信息如表~\ref{table:localEnvironment}和~\ref{table:cloudEnvironment}所示。

\begin{table}[htb]\scriptsize
\centering
\caption{本地环境配置}
\vspace{2mm}
% l - left, r - right, c - center. | means one vertical line
\begin{tabular}{ccc}
\toprule
\textbf{配置项}&\textbf{备注}\\
\midrule
\textbf{CPU}&8核\\ \hline
\textbf{内存}&8G\\ \hline
\textbf{SSD}&512G\\ \hline
\textbf{操作系统}&Windows\\ \hline
\textbf{语言环境支持}&Java、Node.js、Python\\
\bottomrule
\end{tabular}
\label{table:localEnvironment}
\end{table}

\begin{table}[htb]\scriptsize
\centering
\caption{Google Cloud环境配置}
\vspace{2mm}
% l - left, r - right, c - center. | means one vertical line
\begin{tabular}{ccc}
\toprule
\textbf{配置项}&\textbf{备注}\\
\midrule
\textbf{GPU}&NVIDIA Tesla P4\\ \hline
\textbf{内存}&15G\\ \hline
\textbf{SSD}&50G\\ \hline
\textbf{操作系统}&Ubuntu\\ \hline
\textbf{语言环境支持}&Python\\ \hline
\textbf{模型环境支持}&TensorFlow\\
\bottomrule
\end{tabular}
\label{table:cloudEnvironment}
\end{table}

\section{功能测试}

根据功能性需求分析,本章对Fabric智能合约初始模板生成模块、代码智能提示模块、模拟调用及测试模块的功能设计了测试用例,验证各功能是否能在给定输入的情况下按照预期返回结果。同时,各功能的测试代码也为重用插件代码的开发者提供了接口调用的示例。

通过测试用例的设计和测试代码的实现顺利完成了插件各模块的功能测试,测试用例全部通过,测试结果均符合预期,验证了插件各功能实现的正确性,插件可以正常运行且满足用户功能性需求,具有可用性。

\subsection{Fabric智能合约初始模板生成模块测试}

Fabric智能合约初始模板生成模块主要测试能否根据智能合约开发者需求生成符合预期的初始智能合约文件,其测试用例描述如表~\ref{table:tc1}所示。

\begin{table}[htb]\scriptsize
\centering
\caption{初始模板生成测试用例}
\vspace{2mm}
% l - left, r - right, c - center. | means one vertical line
\begin{tabular}{ccc}
\toprule
\textbf{ID}&TC1\\
\midrule
\textbf{名称}&初始模板生成\\ \hline
\textbf{测试目标}&能否根据用户选择成功创建初始智能合约\\ \hline
\textbf{前置条件}&用户当前工作空间存在项目结构\\ \hline
\textbf{输入}&智能合约名称和开发模式\\ \hline
\multirow{3}*{\textbf{测试步骤}}&1.用户输入所要创建智能合约的名称\\
~&2.用户选择智能合约开发模式\\ 
~&3.创建初始智能合约\\ \hline
\multirow{3}*{\textbf{预期结果}}&1.弹出智能合约创建成功提示框,包含路径等信息\\
~&2.生成智能合约文件\\ 
~&3.智能合约内容与相应FTL模板匹配\\
\bottomrule
\end{tabular}
\label{table:tc1}
\end{table}

\subsection{Fabric智能合约代码提示模块测试}

Fabric智能合约代码提示模块主要测试用户在IntelliJ IDE编辑框键入文本时是否具有代码自动补全效果,其测试用例描述如表~\ref{table:tc2}所示。

\begin{table}[htb]\scriptsize
\centering
\caption{代码智能提示测试用例}
\vspace{2mm}
% l - left, r - right, c - center. | means one vertical line
\begin{tabular}{ccc}
\toprule
\textbf{ID}&TC2\\
\midrule
\textbf{名称}&智能合约代码智能提示\\ \hline
\textbf{测试目标}&用户键入文本序列时是否提供代码自动补全选项\\ \hline
\textbf{前置条件}&用户编辑JavaScript文件\\ \hline
\textbf{输入}&IntelliJ IDE编辑框当前键入文本序列\\ \hline
\multirow{2}*{\textbf{测试步骤}}&1.用户键入智能合约代码\\
~&2.用户选中代码自动补全列表的备选项\\ \hline
\multirow{3}*{\textbf{预期结果}}&1.弹出代码自动补全下拉列表\\
~&2.列举模型预测结果集\\ 
~&3.补全用户键入代码序列\\
\bottomrule
\end{tabular}
\label{table:tc2}
\end{table}

IntelliJ平台的测试框架对插件功能的测试提供了很多辅助方法。
在测试代码中,可以通过调用~\texttt{type()}模拟在内存编辑器内键入字符;
通过调用~\texttt{complete()}模拟对代码自动补全列表的请求动作;
通过调用~\texttt{checkRes-} \linebreak ~\texttt{ultByFile()}方法比较预期和实际的操作结果。

插件在IntelliJ IDE中提供实时代码提示的效果很大程度上取决于GPT-2模型训练的效果。
在对GPT-2模型的训练过程中记录了逻辑准确率作为重要的评估指标之一,并将历史数据可视化便于观察模型训练结果,如图~\ref{fig:6.1}所示。

\begin{figure}[htb]
  \centering
  \includegraphics[width=0.6\linewidth]{figure/implement/Accuracy.png}
  \caption{训练集和验证集的逻辑准确率}\label{fig:6.1}
\end{figure}

如上图所示,GPT-2模型训练集和验证集的逻辑准确率在训练第十轮左右趋于平缓,训练停止时验证集的准确率已接近95\%。
由此可见,经过数据采集、数据预处理以及模型训练与评估的反复迭代过程,GPT-2模型的精度得到了改善。

\subsection{Fabric智能合约模拟调用模块测试}

Fabric智能合约模拟调用模块主要测试用户能否调用和测试智能合约接口,其测试用例描述如表~\ref{table:tc3}所示。

\begin{table}[htb]\scriptsize
\centering
\caption{模拟调用测试用例}
\vspace{2mm}
% l - left, r - right, c - center. | means one vertical line
\begin{tabular}{ccc}
\toprule
\textbf{ID}&TC3\\
\midrule
\textbf{名称}&智能合约模拟调用\\ \hline
\textbf{测试目标}&用户能否成功获取并调用智能合约接口\\ \hline
\textbf{前置条件}&存在相应智能合约\\ \hline
\textbf{输入}&智能合约名称、接口名称及参数\\ \hline
\multirow{4}*{\textbf{测试步骤}}&1.打开模拟调用工具窗口\\
~&2.点击按钮获取智能合约接口列表\\ 
~&3.填写智能合约接口对应参数\\
~&4.点击按钮调用智能合约接口\\ \hline
\multirow{2}*{\textbf{预期结果}}&1.显示智能合约接口列表\\
~&2.在工具窗口的自定义控制台显示脚本执行报告\\ 
\bottomrule
\end{tabular}
\label{table:tc3}
\end{table}

\section{性能测试}

性能测试部分主要的目标对象是插件的代码智能提示模块和模拟调用模块。插件可以运行在基于IntelliJ平台开发的IDE中,测试显示插件满足各项性能指标,不会影响IDE其他功能的提供。
表~\ref{table:test}展示了插件各项性能指标及其测试结果。

\begin{table}[htb]\scriptsize
\centering
\caption{性能测试}
\vspace{2mm}
% l - left, r - right, c - center. | means one vertical line
\begin{tabular}{ccc}
\toprule
\textbf{对象}&\textbf{指标}&\textbf{测试结果}\\
\midrule
\textbf{代码智能提示}&可用性&>=99.999\%\\ \hline
\textbf{代码智能提示}&HTTP请求延迟&<1ms\\ \hline
\textbf{插件}&CPU占用率&max<60\%\\ \hline
\textbf{模拟调用}&平均响应时间&<10ms\\
\bottomrule
\end{tabular}
\label{table:test}
\end{table}

\section{本章小结}

本章基于第三章功能性和非功能性需求的分析,对Fabric智能合约初始模板生成、代码智能提示、模拟调用及测试模块的功能进行测试工作。
经过功能测试和性能测试,确保了插件功能的可用性和稳定性,同时插件在IntelliJ IDE中运行的性能满足一定要求。

\chapter{总结与展望}

\section{总结}

区块链进入可编程社会的发展阶段,为金融、监管科技、支付、医疗等多种场景提供了一种可追踪溯源且安全不可篡改的数据存储和交易方式。Hyperledger Fabric针对企业级联盟链实现了隐私性更好且交易速度更快的区块链网络架构。随着该平台应用场景复杂程度的提高,智能合约所承载的业务逻辑体量也逐渐增大。然而,当前主流IDE没有针对智能合约的开发提供良好的支持,导致了其开发效率低下、代码漏洞频发、部署困难等问题。

本文针对上述挑战设计了面向Hyperledger Fabric智能合约的开发插件,该解决方案在无需搭建区块链网络的前提下,提供了一个链下开发智能合约的轻量级辅助工具,降低了智能合约的开发门槛,帮助开发者提升开发和部署效率,同时,对智能合约的代码质量起到了保障作用。

插件基于IntelliJ平台和DevKit SDK开发,实现了Fabric智能合约初始模板生成、代码智能提示、模拟调用及测试三个模块。初始模板生成模块支持用户根据需求和应用场景创建初始智能合约,底层依赖FreeMarker模板引擎和Java IO技术,基于Action系统实现了用户接口层;代码智能提示模块基于自然语言处理的GPT-2模型提供了编写智能合约代码的实时自动补全效果;模拟调用及测试模块通过Hyperledger Fabric提供的ChaincodeMockStub模拟桩和Mocha框架实现了查询智能合约接口,接口模拟调用、调试及测试功能。

本文结合业务背景和研究现状阐述了插件的开发需求,依据需求分析给出了了总体架构设计和模块划分设计,并对各个功能模块详细阐述了设计思路。通过关键代码的展示阐述了插件的实现细节,最后给出了插件的测试流程和结果,说明了插件满足功能性和非功能性需求。

当开发者需要利用IDE完成指定任务而IDE没有针对该需求提供支持时,自定义插件是一种灵活、便捷的途径。利用本文设计并实现的智能合约开发插件可以在智能合约部署到区块链Peer节点之前,保证其语法和业务逻辑的正确性,具有高可用性和高易用性,推动了Hyperledger Fabric开源社区的发展。

\section{进一步工作展望}

本文介绍的插件已经完成了功能的开发和测试,可以部署在远程仓库供智能合约开发者安装使用,但是仍然存在一些可以优化的地方:

\begin{itemize}
    \item 创建初始智能合约的FTL模板是由插件提供的,后续可以考虑基于IntelliJ平台的Action系统提供一个可以自定义FTL模板的用户接口,便于满足更多场景下初始智能合约的创建需求。
    \item Hyperledger Fabric支持使用Node.js、Go、Java等多种语言编写智能合约,插件目前只针对Node.js智能合约的开发,因此各功能模块还需要提供对其他语言的支持。
    \item Fabric代码智能提示模块由于受到GPU等硬件的设备限制选择了基于GPT-2预训练模型的最小版本进行实现,该版本的参数量级为117M,后续可以使用容纳更多参数和解码器堆叠层数的预训练模型,其文本生成的效果会得到很大提升。
    \item 插件支持智能合约接口的模拟调用,其目前的接口调用顺序是与开发者在智能合约中的接口编写顺序相符的,未来可以通过实现更复杂的工具窗口GUI,帮助智能合约开发者简单方便地调整调用接口的顺序。
    \item 插件目前主要关注链下智能合约开发的整体流程,包括创建、编辑、运行、调试及测试,但是未来可以着眼于开发的后续工作,例如对智能合约的打包和部署提供支持。
\end{itemize}

%%%%%%%%%%%%%%%%%%%%%%%%%%%%%%%%%%%%%%%%%%%%%%%%%%%%%%%%%%%%%%%%%%%%%%%%%%%%%%%
% 致谢,应放在结论之后
\begin{acknowledgement}
	白驹过隙,两年的硕士生活即将画上句点,南京大学带给了我无限的激情与收获。
	在这个过程中,我感受到了知识的充盈,也经历了生活的成长。
	感谢我美好的母校,感谢在实验室里的点点滴滴。

	在整个毕业设计和学业的完成过程中,离不开指导教师、学长学姐、同学和家人对我的帮助与支持。

	感谢我的指导教师——邵栋老师和张贺老师。
	从选题初期开始,就耐心地与我探讨,为毕业设计题目的确定带来启示。
	在论文的撰写过程中,老师也会细心指导。
	完成初稿后,老师建议我如何修改与完善。
	老师的辛苦付出,使得整个毕业论文的撰写过程有条不紊,思路清晰。
	依据老师给出的建议,再经过自己的思考并反复修改最终定稿。

	感谢所有带给我知识和帮助的老师们。从对编程懵懵懂懂,成长为可以独立做出一些成果,是他们的悉心指导使我不断充实自己。

	感谢实验室的小伙伴和我的朋友们。在我遇到问题和困难的时候,是他们与我一起探讨,帮助我一起解决,给我提供了很多自己的想法。因为有了他们,生活也变得多姿多彩。

	还要感谢家人们给我的支持。倾听我分享学习、工作和生活上的事,理解我的想法与决定,信任并支持我的选择,带给我关心和爱,他们永远是我最坚强的后盾。

	最后,感谢在百忙之中耐心评阅论文的各位老师!

	在未来的学习与成长的道路上,我也会努力前行,坚持探索,增强实践能力,不负老师和亲人们的殷切期望。最后,祝愿母校人才济济,越来越好!
\end{acknowledgement}




% 参考文献。应放在\backmatter之前。
% 推荐使用BibTeX,若不使用BibTeX时注释掉下面一句。
%\nocite{*}
\bibliography{sample}


% 附录,必须放在参考文献后,backmatter前
%\appendix
%\chapter{附录代码}\label{app:1}
%\section{main函数}
%\begin{lstlisting}[language=C]
%int main()
%{
%	return 0;
%}
%\end{lstlisting}

%%%%%%%%%%%%%%%%%%%%%%%%%%%%%%%%%%%%%%%%%%%%%%%%%%%%%%%%%%%%%%%%%%%%%%%%%%%%%%%
% 书籍附件
\backmatter
%%%%%%%%%%%%%%%%%%%%%%%%%%%%%%%%%%%%%%%%%%%%%%%%%%%%%%%%%%%%%%%%%%%%%%%%%%%%%%%
% 作者简历与科研成果页,应放在backmatter之后
\begin{resume}
% 论文作者身份简介,一句话即可。
\begin{authorinfo}
\noindent 李天歌,女,汉族,1997年7月出生,黑龙江省哈尔滨人。
\end{authorinfo}
% 论文作者教育经历列表,按日期从近到远排列,不包括将要申请的学位。
\begin{education}
\item[2019年9月 --- 2021年6月] 南京大学软件学院 \hfill 硕士
\item[2015年9月 --- 2019年6月] 东北大学软件学院 \hfill 本科
\end{education}
% 论文作者在攻读学位期间所发表的文章的列表,按发表日期从近到远排列。
% 论文作者在攻读学位期间参与的科研课题的列表,按照日期从近到远排列。
\end{resume}

%%%%%%%%%%%%%%%%%%%%%%%%%%%%%%%%%%%%%%%%%%%%%%%%%%%%%%%%%%%%%%%%%%%%%%%%%%%%%%%
% 生成《学位论文出版授权书》页面,应放在最后一页
%\makelicense

%%%%%%%%%%%%%%%%%%%%%%%%%%%%%%%%%%%%%%%%%%%%%%%%%%%%%%%%%%%%%%%%%%%%%%%%%%%%%%%
\end{document}
