\chapter{总结与展望}

\section{总结}

区块链进入可编程社会的发展阶段,为金融、监管科技、支付、医疗等多种场景提供了一种可追踪溯源且安全不可篡改的数据存储和交易方式。Hyperledger Fabric针对企业级联盟链实现了隐私性更好且交易速度更快的区块链网络架构。随着该平台应用场景复杂程度的提高,智能合约所承载的业务逻辑体量也逐渐增大。然而,当前主流IDE没有针对智能合约的开发提供良好的支持,导致了其开发效率低下、代码漏洞频发、部署困难等问题。

本文针对上述挑战设计了面向Hyperledger Fabric智能合约的开发插件,该解决方案在无需搭建区块链网络的前提下,提供了一个链下开发智能合约的轻量级辅助工具,降低了智能合约的开发门槛,帮助开发者提升开发和部署效率,同时,对智能合约的代码质量起到了保障作用。

插件基于IntelliJ平台和DevKit SDK开发,实现了Fabric智能合约初始模板生成、代码智能提示、模拟调用及测试三个模块。初始模板生成模块支持用户根据需求和应用场景创建初始智能合约,底层依赖FreeMarker模板引擎和Java IO技术,基于Action系统实现了用户接口层;代码智能提示模块基于自然语言处理的GPT-2模型提供了编写智能合约代码的实时自动补全效果;模拟调用及测试模块通过Hyperledger Fabric提供的ChaincodeMockStub模拟桩和Mocha框架实现了查询智能合约接口,接口模拟调用、调试及测试功能。

本文结合业务背景和研究现状阐述了插件的开发需求,依据需求分析给出了了总体架构设计和模块划分设计,并对各个功能模块详细阐述了设计思路。通过关键代码的展示阐述了插件的实现细节,最后给出了插件的测试流程和结果,说明了插件满足功能性和非功能性需求。

当开发者需要利用IDE完成指定任务而IDE没有针对该需求提供支持时,自定义插件是一种灵活、便捷的途径。利用本文设计并实现的智能合约开发插件可以在智能合约部署到区块链Peer节点之前,保证其语法和业务逻辑的正确性,具有高可用性和高易用性,推动了Hyperledger Fabric开源社区的发展。

\section{进一步工作展望}

本文介绍的插件已经完成了功能的开发和测试,可以部署在远程仓库供智能合约开发者安装使用,但是仍然存在一些可以优化的地方:

\begin{itemize}
    \item 创建初始智能合约的FTL模板是由插件提供的,后续可以考虑基于IntelliJ平台的Action系统提供一个可以自定义FTL模板的用户接口,便于满足更多场景下初始智能合约的创建需求。
    \item Hyperledger Fabric支持使用Node.js、Go、Java等多种语言编写智能合约,插件目前只针对Node.js智能合约的开发,因此各功能模块还需要提供对其他语言的支持。
    \item Fabric代码智能提示模块由于受到GPU等硬件的设备限制选择了基于GPT-2预训练模型的最小版本进行实现,该版本的参数量级为117M,后续可以使用容纳更多参数和解码器堆叠层数的预训练模型,其文本生成的效果会得到很大提升。
    \item 插件支持智能合约接口的模拟调用,其目前的接口调用顺序是与开发者在智能合约中的接口编写顺序相符的,未来可以通过实现更复杂的工具窗口GUI,帮助智能合约开发者简单方便地调整调用接口的顺序。
    \item 插件目前主要关注链下智能合约开发的整体流程,包括创建、编辑、运行、调试及测试,但是未来可以着眼于开发的后续工作,例如对智能合约的打包和部署提供支持。
\end{itemize}