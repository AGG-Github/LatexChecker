\chapter{插件的需求分析}

\section{功能性需求分析}

\subsection{功能需求概述}

面向Hyperledger Fabric智能合约的开发插件作为辅助区块链开发人员进行智能合约开发的工具,需要满足智能合约开发周期涉及的创建、编写、调用以及测试等功能性需求,插件的用例图如图~\ref{fig:3.1}所示:

\begin{itemize}
  \item 初始代码生成,智能合约开发者需要根据Hyperledger Fabric的使用版本和应用场景选取合适的智能合约初始模板,代码生成模块将为开发者创建一份初始智能合约。
  \item 代码智能提示,开发者在IDE中编写智能合约代码时,需要插件提供实时的代码智能提示,该功能需要补全当前键入的单词序列,并给出备选提示。
  \item 模拟调用及测试,当开发者完成智能合约编写后,需要插件提供工具窗口,展示当前项目中的智能合约及其内部的接口信息,开发者可以通过该窗口完成智能合约的模拟调用并获取测试结果。
\end{itemize}

\begin{figure}[htb]
  \centering
  \includegraphics[width=5in]{figure/requirement/用例图.png}
  \caption{用例图}\label{fig:3.1}
\end{figure}

\subsection{Fabric智能合约初始模板生成模块}

Hyperledger Fabric 1.0正式版本于2017年发布,在IBM、Google等企业的大力支持下,其2.0正式版本于2020年发布.
升级后的Fabric主要增加了以下几点:

\begin{itemize}
  \item 智能合约去中心化管理。
  \item 用于协作和共识的新链码模式。
  \item 外部的链码启动程序。
  \item 新的共识机制。
  \item 新的智能合约开发包。
\end{itemize}

本论文经调研发现,虽然新版本的Hyperledger Fabric具有很多新优势,但是相比1.0版本其稳定性有所下降,而且由于目前新版本发布时间较短,其相关研究和应用程序较少,所以仍然有不少企业在使用1.0版本。

Fabric 1.0版本对智能合约的开发提供了fabric-shim包,fabric-shim是较为底层的智能合约开发包,它封装了Hyperledger Fabric的网络通信协议gRPC,使得开发者可以直接调用链码接口。基于fabric-shim开发的智能合约需要定义一个类,在这个类中需要实现两个预定义的方法,分别是~\texttt{Init()}方法和~\texttt{Invoke()}方法,这两个方法都会接收节点传入的stub作为参数,通过stub可以进行账本的查询、更新等操作。在部署完毕的区块链网络中进行链码初始化时,客户端应用程序会调用~\texttt{Init()}方法,而客户端应用程序对智能合约接口的调用将会转化为对~\texttt{Invoke()}方法的调用。
基于fabric-shim包进行智能合约编写较为简洁易用,适合业务逻辑简单的应用场景,但实现业务逻辑较为复杂应用场景则需要开发者在~\texttt{Invoke()}方法的实现中进行方法路由。

Fabric 2.0版本增加了一种新的链码开发包fabric-contract-api,该开发包提供了对智能合约更高层级的封装,使得开发者可以直接继承Contract类进行智能合约开发,减少了方法路由所产生的繁杂冗余代码。

由于Hyperledger Fabric两个版本各有优势劣势,其提供的智能合约开发包的适用场景也不尽相同,因此插件针对两种智能合约开发方式实现了初始模板生成的功能。插件需要IDE能够弹出提示窗口帮助用户创建初始智能合约,并在用户未创建项目环境等异常情况下给出错误警告和提示信息。

\subsection{Fabric智能合约代码智能提示模块}

随着区块链进入3.0可编程社会阶段,金融、物联网、供应链、医疗等多种领域都开始引入Hyperledger Fabric作为企业联盟链的区块链平台,为其提供数据存储和交易等方面的安全保障。智能合约所承载的业务逻辑逐渐复杂起来,代码体量将愈发庞大。因此,智能合约的编写工作不再适合使用简单的文本编辑器,否则对于区块链开发者而言会增大难度,也会浪费时间精力。

IntelliJ IDE作为当前主流的程序集成开发环境,对项目构建、代码编写、程序编译、运行、调试等需求提供了强大的支持,极大地提高了程序的开发效率。代码编辑器是IntelliJ IDE的重要组件之一,从代码编写角度实现了全面的功能,包括代码自动补全、语法高亮、错误提示等。

因为Hyperledger Fabric智能合约可以使用Java、Go等高级程序语言进行开发,所以开发者适合选择IntelliJ IDEA、GoLand等IDE代替普通的文本编辑器进行代码开发。

运行在Peer节点上的智能合约是客户端应用程序与区块链网络中的账本进行交互的唯一途径。与普通的应用程序相比,智能合约开发者在编码过程中不仅需要学习编程语言的特性和语法规则,还需要考虑智能合约代码的特殊设计以及规范性方面的要求,而这需要开发者熟悉Hyperledger Fabric平台对于智能合约所提供的对外接口。因此,插件需要在IntelliJ IDE代码补全功能的基础之上做进一步的增强效果,针对Hyperledger Fabric两个版本的智能合约编写提供代码智能提示功能。

IntelliJ IDE本身提供的代码补全效果是以键入字符序列时自动弹出的下拉列表的形式展现的。插件需要在开发者键入字符序列的同时,实时地展示代码智能提示,将GPT-2模型预测的Top-k选项加入下拉清单中作为备选提示,开发者可以根据编码目的选择最合适的选项。

插件需要基于IntelliJ平台提供的DevKit SDK开发包进行开发,在不影响IDE性能的前提下,将代码智能提示功能与IDE原有的代码补全效果相融合。

插件针对Hyperledger Fabric智能合约提供的代码智能提示功能可以降低智能合约的开发门槛,提升智能合约的开发效率,与此同时,让开发者将注意力集中在业务逻辑的实现上。

\subsection{Fabric智能合约模拟调用模块}

Hyperledger Fabric在网络搭建层面既支持单机上的单个或多个节点,也支持多机多节点部署。客户端应用程序在调用智能合约接口以对区块链账本进行增删改查等逻辑行为之前,至少需要在本地成功搭建一个单节点的Fabric网络。然而,Hyperledger Fabric的搭建过程较为复杂,而且在Fabric升级到2.0版本后,其平台搭建过程与1.0版本相比产生了较大的区别,智能合约的生命周期也发生了改动。

由于区块链网络的搭建繁杂耗时,且仅针对智能合约的开发而言不是必要条件,因此插件需要提供一个在无需搭建区块链网络的前提下就能对智能合约进行接口模拟调用的入口。

在实际开发环境中,由于开发者缺少一个针对Hyperledger Fabric智能合约的开发工具,因此在网络中部署的智能合约漏洞频发,导致其无法在Peer节点被执行。当智能合约执行报错时,开发者需要搭建并配置Dev模式进行智能合约调试,每一次修改智能合约后,仍然需要重新启动整个Fabric网络,创建通道,将参与交易的Peer节点加入通道,编译、打包、安装并实例化智能合约,开发者需要经历一系列复杂且冗余的操作后,才能重新尝试执行智能合约。
Hyperledger Fabric智能合约部署流程如图~\ref{fig:3.2}所示。

\begin{figure}[htb]
  \centering
  \includegraphics[width=5in]{figure/requirement/智能合约部署流程.png}
  \caption{智能合约部署流程}\label{fig:3.2}
\end{figure}

安全性是区块链技术最关注的因素之一,而网络整体的安全程度很大程度上取决于网络内的节点数量,在多节点的网络内进行智能合约的反复调试和反复部署更加费时费力。

基于IntelliJ平台开发的IDE没有针对Hyperledger Fabric智能合约调试或者测试提供相关的功能,因此,本插件作为一个Hyperledger Fabric智能合约的链下开发工具,需要为开发者提供可以进行智能合约模拟调用的工具窗口,并对测试结果进行展示。

智能合约开发者可以利用本插件的模拟调用及测试功能,在将智能合约发布到区块链网络节点之前就确保其实现和业务逻辑的正确性,既提升了代码质量,又能帮助开发人员节省大量时间,提升开发效率。

\section{非功能性需求分析}

本论文设计并实现的插件需要嵌入到IntelliJ平台的IDE中,所以对其非功能性需求较高。
插件性能需要满足一定标准,在不影响IDE其他功能正常使用的同时,改善开发者使用插件的体验感。
插件的性能、稳定性等重要的非功能性衡量指标如表~\ref{table:requirement}所示:

\begin{table}[htb]\scriptsize
\centering
\caption{非功能性需求列表}
\vspace{2mm}
% l - left, r - right, c - center. | means one vertical line
\begin{tabular}{ccc}
\toprule
\textbf{非功能性需求}&\textbf{目标}&\textbf{优先级}\\
\midrule
\textbf{可扩展性}&支持IntelliJ平台扩展点&高\\ \hline
\textbf{可用性}&良好的界面交互设计&高\\ \hline
\textbf{稳定性}&99.99\%的场景稳定运作&高\\ \hline
\multirow{2}*{\textbf{性能}}&对接口的调用的平均响应时间小于50ms;&\multirow{2}*{高}\\
~&智能合约的代码实时提示展示平均在30ms以内&~\\ \hline
\textbf{可适配性}&支持嵌入多种基于IntelliJ平台开发的IDE&中\\ \hline
\textbf{可部署性}&支持本地构建启动和仓库发布&中\\ \hline
\textbf{可重用性}&具有重用插件扩展点和业务逻辑的能力&中\\ \hline
\textbf{耦合性}&插件与自然语言处理服务解耦&中\\
\bottomrule
\end{tabular}
\label{table:requirement}
\end{table}


插件作为IntelliJ IDE的可插拔式开发辅助工具,需要具备较好的可扩展性。插件的项目结构清晰,且基于IntelliJ平台的DevKit进行开发,可以通过类的继承、配置文件等方式进行功能点或业务逻辑的扩充。

插件需要通过对话框、工具窗口等形式达到良好的可用性,便于开发者的理解和操作,易于上手。

插件需要设计良好的异常处理机制,且稳定地提供NLP服务以维护智能合约代码智能补全的实时效果,具有较高的可靠性。

插件需要进行良好的本地构建和仓库发布,以确保与IntelliJ平台的多种集成开发环境能够兼容适配,便于部署。

IntelliJ平台的插件开发除了可以基于DevKit SDK,还可以基于GitHub Template、Gradle或Maven等构建工具,本插件的功能模块可以快速便捷地迁移到其他方式开发的插件中。

\section{本章小结}

本章从功能性需求和非功能性需求的角度对面向Hyperledger Fabric智能合约的开发插件进行细致的分析。通过用例图展示了插件的功能性需求,并分别对智能合约初始模板生成、代码智能提示、模拟调用及测试等功能进行需求分析;非功能性需求从插件的性能、兼容、有效性等方面考虑并进行阐述。为下一章插件的概要设计做好了铺垫。